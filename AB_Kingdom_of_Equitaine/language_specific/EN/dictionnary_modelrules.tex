% Army Model Rule Names
\newcommand{\theblessing}{The Blessing}
\newcommand{\blessing}{Blessing}
\newcommand{\orison}{Orison}
\newcommand{\orisons}{Orisons}
\newcommand{\sainted}{Sainted}
\newcommand{\knightbanneret}{Knight Banneret}
\newcommand{\ordominister}{Ordo Minister}
\newcommand{\gallantry}{Gallantry}
\newcommand{\daring}{Daring}

\newcommand{\courage}{Courage}
\newcommand{\honesty}{Honesty}
\newcommand{\ordeal}{Ordeal}

\newcommand{\lanceformation}{Lance Formation}

\newcommand{\bastardsword}{Bastard Sword}

\newcommand{\mountsupport}{Mount Support}
\newcommand{\preparedposition}{Prepared Position}
\newcommand{\ordained}{Ordained}
% Army Model Rule Texts

\newcommand{\theblessingef}{%
	Unit profiles in this Army Book contain an additional Characteristic, which corresponds to the units' Aegis Saves, shortened \Aeg{}. This Characteristic is treated as if the unit has the Personal Protection Aegis (X+) written on its profile, where X is the \Aeg{} Characteristic value. Not having an \Aeg{} value does not prevent a unit from being the target of an Aegis modifier. 
}

\newcommand{\orisonsdef}{%
Each Kingdom of Equitaine army has a pool of \blessing{} Tokens that can \textbf{never} contain more than 6 tokens. At the start of step 7 of the Pre-Game Sequence (Spell Selection), add 1 \blessing{} Token per 3000 Army Points to the pool, rounding fractions up.
\removedrule{At the start of each Player Turn,}\newrule{In each Magic Phase, immediately after Siphon the Veil}, \blessing{} Tokens can be discarded. For every discarded \blessing{} Token, choose a single friendly unit and apply one of the following effects until the \removedrule{end of the Player Turn}\newrule{start of the next Magic Phase}:
\begin{itemize}
	\item Orison of Shielding: The unit gains \aegis{} (5+).
	\item Orison of Striking: Model parts without \harnessed{} in the unit gain +1 to hit with Close Combat Attacks.
	\item Orison of Discipline: The unit's Discipline is \textbf{set} to 9.
\end{itemize}
A single unit can only be the target of one Orison per Player Turn, unless specifically stated otherwise.
}

\newcommand{\sainteddef}{%
	Model parts without \harnessed{} gain  \textbf{\fearless{}}  and +1 Attack Value.
	In addition, the model is \textbf{always} under the effect of  Orison of Shielding, Orison of Striking, and Orison of Discipline.  This does not prevent the model’s unit from being the target of an \orison{}, but the model does not benefit from this additional Orison.%
}

\newcommand{\knightbanneretdef}{%
	The model gains the following rules:
	\begin{itemize}
		\item The model gains +1 Health Point, up to a maximum of 3.
		\item The model may take a single Banner Enchantment from this Army Book, for which it is considered to have a Special Item allowance with no limit. 
		\item When calculating Combat Score, the model adds +1 to its side’s Combat Score.
	\end{itemize}
}

\newcommand{\ordoministerdef}{%
	 At the start of each friendly Magic Phase, each unit containing one or more models with \ordominister{} may remove a single token from the \blessing{} Token pool. If so, the unit, or a model inside the  unit, may Raise 1 Health Point. 
}

\newcommand{\gallantrydef}{%
	During Army List creation, the unit gains a \gallantry{} value that corresponds to the value stated in brackets (X). Multiple instances of \gallantry{} (X) in the same unit do not stack. The sum of the \gallantry{} values of all units on the Army List is restricted to 1 per 650 Army Points, rounding fractions up. %
}

\newcommand{\daringdef}{
	Units with more than half of their models with \daring{} cannot voluntarily declare Flee as a Charge Reaction and \textbf{must} reroll failed Panic Tests.
}

\newcommand{\couragedef}{%
	The model gains \textbf{\aegis{} (5+)} with the following restriction: The effect can only be used against wounds against which the model cannot take or would automatically fail its  Armour Save. \newline
	Units with more than half of their models with \courage{} ignore friendly units consisting entirely of models with \ordeal{} for the purpose of Panic Tests.%
}

\newcommand{\honestydef}{%
	The model gains \textbf{\aegis{} (5+, against \magicalattacks{})}.%
}

\newcommand{\ordealdef}{%
	The model gains \textbf{\aegis{} (5+)} while its unit is Engaged in the same Combat as at least one other friendly unit.
}

\newcommand{\lanceformationdef}{%
	The model gains \textbf{\fightinextrarank{}}. In addition, if the model is Standard, it gains \extrasupport{} (2).  If more than half of a unit’s models have Lance Formation and the unit is 3 or 4 models wide, it counts as being in Line Formation and only needs to be 3 models wide in order to form Full Ranks.
}

\newcommand{\bastardsworddef}{%
	Hand Weapon.  Attacks made with  a \bastardsword{} gain +1 Strength and \textbf{\devastatingcharge{} (+1 AP)}.
}

\newcommand{\mountsupportdef}{
The model part ignores Harnessed for the purpose of Supporting Attacks.
}

\newcommand{\preparedpositiondef}{
When deploying the unit, you may place a Wall Terrain Feature \newrule{fully within \distance{1} of the unit's Front Facing}\removedrule{with its centre within \distance{2} of the unit} but not in contact with any other Terrain Feature except Open Terrain. This Wall is up to \distance{1} deep \newrule{and its length cannot exceed the width of the unit, up to a maximum of \distance{12}.}\removedrule{and up to \distance{8} wide and} \newrule{It} follows the normal rules for Walls, with the exception that it contributes to Soft Cover instead of Hard Cover.

}
\newcommand{\ordaineddef}{%
At the start of step 7 of the Pre-Game Sequence (Spell Selection), add 1 \blessing{} Token to your \blessing{} Token pool for each model with \ordained{} on your Army List.
}