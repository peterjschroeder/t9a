
%%%%%%%%%%%%%%%%%%%%%%%%%%%%%%%%%%%
%%%%%%%%%%%%%%%%% Romuald's libs



\usepackage[margin=0mm]{geometry}

\usepackage[utf8]{inputenc}
\usepackage[T1]{fontenc} 
\usepackage[french]{babel}

%\usepackage{lmodern}
\usepackage[light]{anttor} %condensed, light
%font http://www.tug.dk/FontCatalogue/antykwatorunska/



\usepackage{microtype}
% microtype – Subliminal refinements towards typographical perfection
% https://www.ctan.org/pkg/microtype?lang=en
\sloppy

\usepackage{graphicx}
\usepackage{color}

\usepackage{tikz}
\usetikzlibrary{patterns}
\usetikzlibrary{shadows}
\usetikzlibrary{fadings}
\usetikzlibrary{svg.path}

\usepackage{url}

\pagestyle{empty}

\setlength{\parindent}{0pt}
% equivalent to global noi,dent

\definecolor{titlebg}{RGB}{30,30,30}
% title backbroung color

\definecolor{popbg}{RGB}{0,100,200}   %~blue
\definecolor{hardbg}{RGB}{80,180,0}   %~light green
\definecolor{softbg}{RGB}{200,50,50}  %~red
\definecolor{langbg}{RGB}{100,50,30}  %~maroon
\definecolor{theorybg}{RGB}{180,50,150}  %~pink
\definecolor{gamebg}{RGB}{200,140,20} %~ocre
\definecolor{webbg}{RGB}{40,90,20}  %~dark green

\definecolor{manbg}{RGB}{70,70,70} %~grey
\definecolor{univbg}{RGB}{70,70,70}
\definecolor{rulesbg}{RGB}{70,70,70}

% categories background colors

\definecolor{contentbg}{RGB}{255,255,255}
%\definecolor{contentbg}{RGB}{250,250,245}
% main (text) content color



%%%%%%%%%%%%%%%%%%%%%%%%%%%%%%%%
%%%%%%%%%%%%%%%% Arvid's libs





%   VORAUSGESETZTE LIBS
%   ------------------

%   Ränder des Dokuments anpassen
\usepackage[margin=6mm,top=5mm]{geometry}

%   Schriftart der auf den Karten eingesetzten Texte
\usepackage{anttor}

%   UTF-8 Encoding der TeX-Dateien
\usepackage[utf8]{inputenc}

%   deutsches Sprachpaket
\usepackage[german]{babel}

%   optischer Randausgleich
\usepackage{microtype}

%   Einbinden von Grafiken
\usepackage{graphicx}

%   Definieren und Verwenden von Farben
\usepackage{color}

%   TikZ zum "Malen" von Grafiken, in diesem Falle für die Karten
\usepackage{tikz}
\usetikzlibrary{patterns}
\usetikzlibrary{shadows}

%   Symbole dazuladen; Verwendung \ding{<nummer>}
\usepackage{pifont}
%   weitere Symbole
\usepackage{fourier-orns}