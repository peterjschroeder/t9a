
\newnarrativescenario{12}{In the Mountains}
\label{IntheMountains}

\flufffont{Dwarves don't usually appreciate when foreigners march on their territory without authorization. I was merely passing by, bringing my troops as a reinforcement for a most important war. I knew I would have been lucky if the dwarves had let me go through the valley, but walking around their territory would have taken more time than I can afford. Now I just have make most of my troops reach the other side.}

\subsection*{Deployment}

In this scenario, one player takes on the role of the Attacker while the other player is the Defender. Use a standard \distance{72}\timess{}\distance{48} board. Randomly choose one short Board Edge to be the Attacker's Edge. The opposite short Board Edge is the Defender's Edge.

Place two Hills, one along each long Board Edge, as long as the Board Edges and with a width of \distance{10}. The long edges of the Hills facing towards the centre of the board are called Hill Lines. Put three 1\timess\distance{12} Impassable Terrains on each Hill Line (six in total), parallel to the Hill Lines, and with their centres on the Hill Lines at \distance{6}, \distance{36}, and \distance{66} from the Attacker's Edge. Then move each of them \distance{2D6} along the Hill Lines, randomising in which direction (if it can't be moved there, don't move it at all).

Finally, add three Ruins on the Centre Line at \distance{12}, \distance{36}, and \distance{60} from the Attacker's Edge, and move them \distance{2D6} in a random direction.

The Attacker's Deployment Zone is an area less than \distance{14} away from the Centre Line and less than \distance{20} away from the Attacker's Edge. The Defender's Deployment Zone correspond to both Hills, at least \distance{24} away from the Attacker's Edge.

\newcommand{\figTwelveDefender}{{\Largefontsize{Defender}}}
\newcommand{\figTwelveAttacker}{{\Largefontsize{Attacker}}}
\newcommand{\repeatthreetimes}{repeat 3\timess{}}
\newcommand{\repeatsixtimes}{repeat 6\timess{}}
\printmapmountains{pics/deployment_12_in_the_mountains.pdf_tex}

\newpage
\subsection*{Pre-Game Set-up}

The Defender's army must be Infernal Dwarves or Dwarven Holds. Its Army Points are limited to 3500 pts, but the limit of Hail of the Gods is increased to max. \SI{45}{\percent}, Barrage to max. \SI{30}{\percent}, Clan's Thunder to max. \SI{50}{\percent}, and Engines of War to max. \SI{30}{\percent}. The Attacker's army can be from any Army Book and up to 4500 pts, but units with Fly are limited to max. \SI{25}{\percent}.

The Deployment Phase Sequence is slightly modified. At first the Defender deploys their units with War Machine, plus one unit for every unit with Scout the Attacker has in their Army List (regardless if these units use Special Deployment or not). Then the Attacker deploys their entire army, including their units with Scout, but not including units using other Special Deployment such as Ambush. The units must be deployed within \distance{48} of the Attacker's Edge. Finally, the Defender deploys the rest of their army and their Scouts as usual.

\subsection*{Winning the Scenario}

The game ends after 7 Game Turns. Count the Victory Points as usual. On top of that, when an Attacker's unit finishes a move within \distance{10} of the Defender's Edge, and is not Engaged in Combat nor Fleeing, the Attacker can remove this unit from the game. The Point Costs of all units removed this way count towards the Attacker's total Victory Points. Moreover, all of the Attacker's units that are farther than \distance{10} of the Defender's Edge at the end of the game award their full Point Costs to the Defender, as if they had been removed as casualties.
