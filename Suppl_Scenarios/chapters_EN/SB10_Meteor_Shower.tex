
\newgamingscenario{2}{2}{Meteor Shower}
\label{MeteorShower}

\flufffont{The area has been showered with meteors containing valuable Darkstone. This material has strange properties, but it just might be used to your advantage as you collect the meteor remains.}

\subsection*{Deployment}

Standard Deployment Type: Counterthrust.

\subsection*{Pre-Game Set-up}

After step 8 of the Pre-Game Sequence (immediately before the Deployment Phase), place 5 meteor markers with their centres evenly along the Centre Line (each \distance{12} apart). Then move each marker D6\timess{}\distance{2} in a random direction. If a \result{6} is rolled, the marker is not moved at all. Finally, adjust the position of each marker by the minimal amount until it's centre is \distance{1} clear of all Impassable Terrain if necessary.

\printmap{pics/deployment_2_2_meteor_shower.pdf_tex}

\subsection*{Scenario Special Rules}

At the start of each of your Player Turns, except for the first, any of your units may pick up meteor markers whose centre they are in contact with. Remove the marker from the Battlefield: the unit is now carrying the marker. Units carrying a marker with less than 3 Full Ranks cannot perform March Moves.

If a unit that is carrying markers is destroyed or loses Scoring, the opponent must immediately place all markers carried by this unit with their centres on points within \distance{3} of it. These points cannot be within \distance{1} of Impassable Terrain or the centres of other meteor markers, but can be inside a unit.

Non-Bound Spells cast by models in units carrying one or more meteor markers must reroll all Magic Dice that result in a \result{1}. If the Caster would do so normally, it also gains a +1 Casting Modifier.

\subsection*{Winning the Secondary Objective}

At the end of the game, the player whose Scoring Units carry the most meteor markers wins this Secondary Objective. Note that non-Scoring units may pick up markers but do not count towards winning this Secondary Objective.
