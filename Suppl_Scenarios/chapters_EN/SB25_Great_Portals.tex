
\newnarrativescenario{3}{Great Portals of the Barren Mountains}
\label{GreatPortalsoftheBarrenMountains}

\flufffont{High up in the Barren Mountains, there exist two portals, one on each side of a narrow valley. No one knows who built them or their original purpose, though their value is obvious. Beware though, for the balance of the portals' energy must be maintained.}

\subsection*{Deployment}

The Deployment Zones are referred as Red and Blue. Note their placement relative to the Portals in the figure below. 

\newcommand{\reddeploymentzone}{Red Deployment Zone}
\newcommand{\bluedeploymentzone}{Blue Deployment Zone}

\printmapportals{pics/deployment_3_the_portals_of_the_barren_mountains.pdf_tex}

\subsection*{Pre-Game Set-up}

Place the Portals in two diagonally opposite corners of the board. The bases should be triangular in shape with sides of approximately \distance{18\timess{}8}, assuming a \distance{72\timess{}48} board.

\subsection*{Scenario Special Rules}

There are two kinds of Ionisation Tokens, Red and Blue. Units that receive tokens are referred to as Ionised Units and gain \textbf{Electric Charge} (see below). A unit can only have a single Ionisation Token.

During step 7 of the Deployment Phase Sequence (after moving Vanguarding units), the players take turns marking 3 of their units each with Ionisation Tokens of the same colour as their Deployment Zone, starting with the player who finished deploying first.

\subsubsection*{Electric Charge\titleruletype{\attackattribute}}

When Charging, Ionised Units have special interactions:
\begin{itemize}%					
	\item Charge Range rolls against units of opposite Ionisation are subject to Maximised Roll. If such a Charge is successful, the owner of the Charging unit can choose to immediately switch the Ionisation Tokens between the Charging unit and the Charged unit (once the Charge Move is completed).
	\item Charge Range rolls against units of equal Ionisation are subject to Minimised Roll.
\end{itemize}
	
\subsubsection*{Portal Jumping}

When a unit ends an Advance Move fully within \distance{5} of a Portal, it may choose to either:
\begin{itemize}
	\item Enter it, unless is has an Ionisation of the same colour as the Portal
	\item Lose its Ionisation, if it is Ionized
\end{itemize}

If the unit enters the Portal, it is removed from the Battlefield and then placed back in the same formation, facing any direction, following the Unit Spacing rule, and fully within \distance{5} of the other Portal. If this is impossible, the unit may not enter the Portal. If the teleported unit was not Ionised, it gains an Ionisation Token of the colour of the Portal it exited (e.g. exiting through a Red Portal gives a Red Ionisation).

\subsection*{Winning the Scenario}

At the end of each Player Turn, starting on Game Turn 3, the Active Player may choose to discard Ionisation Tokens from units within \distance{6} of the centre of the board. Discarded tokens are placed in a discard pool at the side of the board. For each discarded Ionisation Token, the player gains a Victory Counter. If the discard pool (which is charged by both players) contains an equal number of Red and Blue Ionisation Tokens after this is done, the player gains an additional Victory Counter.

At the end of the game, the player with the most Victory Counters wins the game.
