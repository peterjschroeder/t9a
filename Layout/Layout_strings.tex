
%%% String manipulation %%%

\def\removespaces#1{\zap@space#1 \@empty}

\pdfstringdefDisableCommands{\def\textcolor#1{}}
\pdfstringdefDisableCommands{\def\newline{}}
\pdfstringdefDisableCommands{\def\removedrule{}}
\pdfstringdefDisableCommands{\def\removedreworded{}}
\pdfstringdefDisableCommands{\def\void{}}
\pdfstringdefDisableCommands{\def\guidingversion#1{}}
\pdfstringdefDisableCommands{\def\refsymbolB{}}
\pdfstringdefDisableCommands{\def\refsymbolBbis{}}
\pdfstringdefDisableCommands{\def\refsB{}}
\pdfstringdefDisableCommands{\def\refsBbis{}}
\pdfstringdefDisableCommands{\def\refsymbolC{}}
\pdfstringdefDisableCommands{\def\refsymbolCbis{}}
\pdfstringdefDisableCommands{\def\refsC{}}
\pdfstringdefDisableCommands{\def\refsCbis{}}
\pdfstringdefDisableCommands{\def\base#1{}}
\pdfstringdefDisableCommands{\def\boosted#1{}}
\pdfstringdefDisableCommands{\def\specialboosted#1{}}
\pdfstringdefDisableCommands{\def\nth#1{}}

\newcommand{\substitute}[3]{%
  \protected@edef\sub@temp{#1}%
  \saveexpandmode%
  \expandarg\StrSubstitute{\sub@temp}{#2}{#3}[#1]%
  \restoreexpandmode%
}

\newcommand{\pdfsubstitute}[3]{%
	% #1 is the string in which we need to substitute something
	% #2 need to be replaced by #3
	\pdfstringdef\specialcharcode{a#2}%
	\pdfstringdef\tempcharcode{a}%
	\substitute\specialcharcode{\tempcharcode}{}%
	\pdfstringdef\replacementcharcode{a#3}%
	\substitute\replacementcharcode{\tempcharcode}{}%
	\substitute{#1}{\specialcharcode}{\replacementcharcode}%
}

\newcommand{\splitatinf}[3]{%
  \protected@edef\split@temp{#1}%
  \saveexpandmode%
  \expandarg\StrCut{\split@temp}{<}#2#3%
  \restoreexpandmode%
}

\newcommand{\splitatstickto}[3]{%
  \protected@edef\split@temp{#1}%
  \saveexpandmode%
  \expandarg\StrCut{\split@temp}{<STICKTO<}#2#3%
  \restoreexpandmode%
}

\newcommand{\splitatequal}[3]{%
  \protected@edef\split@temp{#1}%
  \saveexpandmode%
  \expandarg\StrCut{\split@temp}{=}#2#3%
  \restoreexpandmode%
}


%%% String tests %%%

\newcommand{\ifsubstring}[4]{%
\protected@edef\split@temp{#1}%
\protected@edef\split@tempbis{#2}%
\saveexpandmode%
\expandarg\IfSubStr{\split@temp}{\split@tempbis}{#3}{#4}%
\restoreexpandmode%
}

\newcommand{\isitoneornot}[3]{%
% First step is to remove spaces if there are some
\def\numberwithoutspaces{\expandafter\removespaces\expandafter{#1}}%
% Next step is getting rid of formatting if there are any (bold, color, ...)
\pdfstringdef\cleannumber{\numberwithoutspaces}%
%Defining 1 in \pdfstringdef terms (it will add \376\377\000 before usually - unicode identifier)
\pdfstringdef\numberone{1}%
% Now we can try if it is 1 or not
\ifsubstring{\numberone}{\cleannumber}{#2}{#3}%
}

\newcommand{\isthereaplusornot}[3]{%
\ifsubstring{#1}{+}{#2}{#3}%
}
