
\part{Model Rules}
\idx[main=y]{Model Rules}\label{model_rules}

\RBbmc

\hypertarget{specialrulestable}{}
	\specialruletablesubtitle{universal_rules}{Universal Rules}
	\specialruletableentry{ambush}{Ambush}
	\specialruletableentry{attached}{Attached}
	\specialruletableentry{bsb}{Battle Standard Bearer}
	\specialruletableentry{bodyguard}{Bodyguard}
	\specialruletableentry{channel}{Channel}
	\specialruletableentry{chariot}{Chariot}
	\specialruletableentry{commanding_presence}{Commanding Presence}
	\specialruletableentry{deafening_clamour}{Deafening Clamour}
	\specialruletableentry{exclusive}{Exclusive}
	\specialruletableentry{engineer}{Engineer}
	\specialruletableentry{fear}{Fear}
	\specialruletableentry{fearless}{Fearless}
	\specialruletableentry{feigned_flight}{Feigned Flight}
	\specialruletableentry{fly}{Fly}
	\specialruletableentry{frenzy}{Frenzy}
	\specialruletableentry{front_rank}{Front Rank}
	\specialruletableentry{ghost_step}{Ghost Step}
	\specialruletableentry{hidden}{Hidden}
	\specialruletableentry{hold_the_line}{Hold the Line}
	\specialruletableentry{insignificant}{Insignificant}
	\specialruletableentry{light_troops}{Light Troops}
	\specialruletableentry{magic_resistance}{Magic Resistance}
	\specialruletableentry{massive_bulk}{Massive Bulk}
	\specialruletableentry{maximised}{Maximised}
	\specialruletableentry{minimised}{Minimised}
	\specialruletableentry{not_a_leader}{Not a Leader}
	\specialruletableentry{protean_magic}{Protean Magic}
	\specialruletableentry{rally_around_the_flag}{Rally Around the Flag}
	\specialruletableentry{random_movement}{Random Movement}
	\specialruletableentry{scoring}{Scoring}
	\specialruletableentry{scout}{Scout}
	\specialruletableentry{skirmisher}{Skirmisher}
	\specialruletableentry{special_ambush}{Special Ambush}
	\specialruletableentry{stand_behind}{Stand Behind}
	\specialruletableentry{strider}{Strider}
	\specialruletableentry{stubborn}{Stubborn}
	\specialruletableentry{supernal}{Supernal}
	\specialruletableentry{swift_reform}{Swift Reform}
	\specialruletableentry{swiftstride}{Swiftstride}
	\specialruletableentry{tall}{Tall}
	\specialruletableentry{terror}{Terror}
	\specialruletableentry{towering_presence}{Towering Presence}
	\specialruletableentry{unbreakable}{Unbreakable}
	\specialruletableentry{undead}{Undead}
	\specialruletableentry{unstable}{Unstable}
	\specialruletableentry{vanguard}{Vanguard}
	\specialruletableentry{war_machine}{War Machine}
	\specialruletableentry{war_platform}{War Platform}
	\specialruletableentry{wizard_apprentice}{Wizard Apprentice}
	\specialruletableentry{wizard_adept}{Wizard Adept}
	\specialruletableentry{wizard_master}{Wizard Master}
	\specialruletableentry{wizard_conclave}{Wizard Conclave}

	\specialruletablesubtitle{characters}{Character}
	\specialruletableentry{make_way}{Make Way}

	\specialruletablesubtitle{command_group}{Command Group}
	\specialruletableentry{champion}{Champion}
	\specialruletableentry{first_among_equals}{First Among Equals}
	\specialruletableentry{ordering_the_charge}{Ordering the Charge}
	\specialruletableentry{musician}{Musician}
	\specialruletableentry{march_to_the_beat}{March to the Beat}
	\specialruletableentry{standard_bearer}{Standard Bearer}
	\specialruletableentry{combat_bonus_CG}{Combat Bonus}

	\specialruletablesubtitle{personal_protections}{Personal Protections}
	\specialruletableentry{aegis}{Aegis}
	\specialruletableentry{cannot_be_stomped}{Cannot be Stomped}
	\specialruletableentry{distracting}{Distracting}
	\specialruletableentry{flammable}{Flammable}
	\specialruletableentry{fortitude}{Fortitude}
	\specialruletableentry{hard_target}{Hard Target}
	\specialruletableentry{immune}{Immune}
	\specialruletableentry{parry}{Parry}

	\specialruletablesubtitle{armour_equipment}{Armour Equipment}

	\specialruletablesubtitle{close_combat_weapons}{Close Combat Weapons}

	\specialruletablesubtitle{shooting_weapons}{Shooting Weapons}

	\specialruletablesubtitle{artillery_weapons}{Artillery Weapons}
	\specialruletableentry{cannon}{Cannon}
	\specialruletableentry{catapult}{Catapult}
	\specialruletableentry{flamethrower}{Flamethrower}
	\specialruletableentry{volley_gun}{Volley Gun}
	\specialruletableentry{the_misfire_table}{The Misfire Table}

	\specialruletablesubtitle{attack_attributes}{Attack Attributes}
	\specialruletableentry{accurate}{Accurate}
	\specialruletableentry{area_attack}{Area Attack}
	\specialruletableentry{battle_focus}{Battle Focus}
	\specialruletableentry{crush_attack}{Crush Attack}
	\specialruletableentry{devastating_charge}{Devastating Charge}
	\specialruletableentry{extra_support}{Extra Support}
	\specialruletableentry{divine_attacks}{Divine Attacks}
	\specialruletableentry{fight_in_extra_rank}{Fight in Extra Rank}
	\specialruletableentry{flaming_attacks}{Flaming Attacks}
	\specialruletableentry{harnessed}{Harnessed}
	\specialruletableentry{hatred}{Hatred}
	\specialruletableentry{inanimate}{Inanimate}
	\specialruletableentry{lethal_strike}{Lethal Strike}
	\specialruletableentry{lightning_reflexes}{Lightning Reflexes}
	\specialruletableentry{magical_attacks}{Magical Attacks}
	\specialruletableentry{march_and_shoot}{March and Shoot}
	\specialruletableentry{move_or_fire}{Move or Fire}
	\specialruletableentry{multiple_wounds}{Multiple Wounds}
	\specialruletableentry{poison_attacks}{Poison Attacks}
	\specialruletableentry{quick_to_fire}{Quick to Fire}
	\specialruletableentry{rage}{Rage}
	\specialruletableentry{reload}{Reload!}
	\specialruletableentry{shoot_in_extra_rank}{Shoot in Extra Rank}
	\specialruletableentry{steady_aim}{Steady Aim}
	\specialruletableentry{toxic_attacks}{Toxic Attacks}
	\specialruletableentry{twohanded}{Two-Handed}
	\specialruletableentry{unwieldy}{Unwieldy}
	\specialruletableentry{volley_fire}{Volley Fire}
	\specialruletableentry{weapon_master}{Weapon Master}

	\specialruletablesubtitle{special_attacks}{Special Attacks}
	\specialruletableentry{breath_attack}{Breath Attack}
	\specialruletableentry{grind_attacks}{Grind Attacks}
	\specialruletableentry{impact_hits}{Impact Hits}
	\specialruletableentry{stomp_attacks}{Stomp Attacks}
	\specialruletableentry{sweeping_attack}{Sweeping Attack}

\RBemc

\newpage

\RBbmc

Model Rules are rules that are applied to individual models or model parts, as described in their unit entry. They are divided into the following categories: Universal Rules, Character, Personal Protections, Armour Equipment, Weapons, Attack Attributes, and Special Attacks.

\paragraph{Duplicated Model Rules}
\idx[main=y]{Duplicated Model Rules}

Sometimes a model or model part may have the same Model Rule more than once, for example when a model gains a Model Rule during the game that it already had before. In this case, the effects of the duplicated Model Rule do not stack and do not offer any additional benefit, unless specifically stated otherwise.

If the duplicated Model Rule has different values in brackets (X), use the highest value.

If X is the result of a dice roll, you may instead choose which version to use (before rolling any dice).

If X is not a value, the Model Rules are not considered to be duplicates of the same Model Rule and both rules are applied (e.g. Hatred (against Infantry) and Hatred (against Cavalry) are considered two different Model Rules, so both effects are applied).

\section{Universal Rules}
\idx[main=y]{\universalrules}\label{universal_rules}

If at least one model part has a Universal Rule, the entire Multipart Model is affected by it.

For example, in case of a Character with the \hyperref[strider]{Strider} Universal Rule on a Character mount without this Universal Rule, all model parts of the Multipart Model (Character and mount) benefit from Strider.

\subsection{Conditional Application}
\label{conditional_application_of_rules}
Universal Rules may have the range of their effects modified:
\begin{itemize}
	\item If a range is stated in brackets (X) (e.g. \hyperref[rally_around_the_flag]{\rallyaroundtheflag\ (\distance{18})}), X replaces the Universal Rule's default range.
	\item If X is given as a modifier (e.g. \hyperref[commanding_presence]{\commandingpresence\ (\distance{+6})}), the Universal Rule gains this as a modifier to its range. If the model doesn’t have the Univeral Rule, the modifier has no effect and this instance of the Universal Rule is ignored.
	\item If the brackets include a maximum value (e.g. \hyperref[commanding_presence]{\commandingpresence\ (\distance{+6}, max. 18")}), the Universal Rule's range cannot be increased to better than the maximum value given in brackets.
\end{itemize}
For certain Universal Rules, only specific friendly models may benefit from them, which are then stated in brackets. There may already be some piece of information relative to the rule specified between brackets, as in \hyperref[commanding_presence]{\commandingpresence\ (6")}. In this case, the models that can benefit from the rule are written in the same brackets, after a comma. This can e.g. be models with a given Model Rule or all models in a unit containing a model with a certain Model Rule, like \hyperref[commanding_presence]{\commandingpresence\ (\distance{6}, \insignificant)}.

\subsection{List of Universal Rules}
\label{list_of_universal_rules}

\subsubsection{Ambush}
\idx{Special Deployment}\idx[main=y]{\ambush}\label{ambush}

Units with Ambush may be deployed using Special Deployment rules. All units that will be deployed using the Ambush rule must be declared at step 8 of the \hyperref[the_pre_game_sequence]{Pre-Game Sequence} (after Spell Selection), starting with the player that chose their Deployment Zone. Deploy your army as usual, but without the Ambushing units. Starting with your Player Turn 2, immediately after step 2 of each friendly \hyperref[the_movement_phase_sequence]{Movement Phase Sequence} (after moving units with \hyperref[random_movement]{Random Movement}), roll a dice for each of your Ambushing units. After rolling for all Ambushing units, all units that rolled 3+ enter the Battlefield from any Board Edge. Place the arriving units with their Rear Facing in contact and aligned with the Board Edge.

Ambushers are subject to the following rules and restrictions:
\begin{itemize}
	\item Ambushing models can neither March Move during the Movement Phase in which they arrive, nor can they voluntarily end that Movement Phase farther away from the Board Edge that they arrived from than their March Rate.
	\item For the purpose of shooting, Ambushing models count as having moved during the Player Turn they arrive on the Battlefield.
	\item If an Ambushing unit has not entered the Battlefield before the end of the game (e.g. due to failing all its 3+ rolls), the unit counts as destroyed.
	\item An Ambushing unit that enters the Battlefield on Game Turn 4 or later loses \hyperref[scoring]{Scoring}.
	\item An Ambushing Character may Ambush within an Ambushing unit that it is allowed to join (declare this when declaring which units are Ambushing). Roll only one dice for the Combined Unit.
\end{itemize}

\subsubsection{Attached}
\idx[main=y]{\attached}\label{attached}

The model \textbf{must} be deployed in a unit. The model can \textbf{never} voluntarily leave its unit.

\subsubsection{Battle Standard Bearer -- \oneofakind{}}
\idx[main=y]{Battle Standard Bearer}\label{bsb}

An army may only include a single Battle Standard Bearer. The model gains \hyperref[rally_around_the_flag]{\textbf{Rally Around the Flag}} and \hyperref[not_a_leader]{\textbf{Not a Leader}}. If the model has the option to buy Special Items, it is allowed to buy up to two Banner Enchantments.

\columnbreak

\subsubsection{Bodyguard (X)}
\idx[main=y]{\bodyguard{}}\label{bodyguard}

While a Character is joined to a unit in which at least one model has Bodyguard, that Character gains \hyperref[stubborn]{\textbf{Stubborn}}. When Characters or Character types are stated in brackets, Bodyguard only works for the specified Characters or Character types.

\subsubsection{Channel (X)}
\idx[main=y]{\channel{}}\label{channel}

During step 3 of the \hyperref[magic_phase_sequence]{Magic Phase Sequence}, each of the Active Player's models with Channel may add X Veil Tokens to its owner's Veil Token pool. This Universal Rule is cumulative, adding the X of each instance of Channel to the model's total Channel value (e.g. a model with Channel (1) and Channel (2) is treated like a model with Channel (3)).

\subsubsection{Chariot}
\idx[main=y]{\chariot}\label{chariot}

The model gains \textbf{Exclusive (Chariot)}, unless specifically stated otherwise, and it \textbf{must} roll an additional D6 when taking \hyperref[dangerous_terrain]{Dangerous Terrain Tests}.

\subsubsection{Commanding Presence}
\idx[main=y]{\commandingpresence{}}\label{commanding_presence}

The Discipline of all models in units, including Fleeing units, within \distance{12} of a friendly non-Fleeing model with Commanding Presence may be \textbf{set} to the Discipline value of that model. This ability follows the normal rules for \totalref{values_set_to_a_fixed_number}, meaning that effects modifying the Discipline of the model with Commanding Presence are applied before setting the recipient model's Discipline to that value; this value may then be further modified, yet any Discipline modifier, including Fear, regardless of ifs source, is only applied once for determining a model's Discipline value.

\subsubsection{Deafening Clamour}
\idx[main=y]{\deafeningclamour{}}\label{deafening_clamour}

The model is a Musician. The range of the model’s March to the Beat, and to enemy units that are required to take a March Test due to the model's unit, are both extended to \distance{18}.

\subsubsection{Engineer (X+)}
\idx[main=y]{\engineer{}}\label{engineer}

Once per Shooting Phase, an unengaged Engineer may select a single \hyperref[war_machine]{War Machine} within \distance{6} that has not fired yet during this Shooting Phase to gain the following effects:

\begin{itemize}
	\item \textbf{Set} the Aim of one of the War Machine's Artillery Weapons to the value given in brackets (X+).
	\item You may reroll the roll on the \hyperref[the_misfire_table]{Misfire Table}.
	\item You may reroll the dice (all of them or none) for determining the number of hits of a \hyperref[flamethrower]{Fla\-me\-thrower} Artillery Weapon.
\end{itemize}

The effects last until the end of the Shooting Phase.

\subsubsection{Exclusive (X)}
\idx[main=y]{\exclusive{}}\label{exclusive}

Characters with Exclusive (X) can only join units that contain at least one model with (X), and units consisting entirely of models with Exclusive (X) can only be joined by models with Exclusive (X), where X specifies e.g. certain unit names, Model Rules, or upgrades. Models with Exclusive without any brackets can never join units or be joined by other models.

\subsubsection{Fear}
\idx[main=y]{\fear}\label{fear}

Units in base contact with one or more enemy models with Fear suffer \minuss{}1 Discipline. At the start of each Round of Combat, such units must take a \hyperref[discipline_tests]{Discipline Test}, called a Fear Test. If this test is failed, the models in the unit are \hyperref[shaken]{Shaken} and Close Combat Attacks made by models in the unit suffer \minuss{}1 to hit, while Close Combat Attacks allocated towards models in the unit gain +1 to hit. These effects apply until the end of the Round of Combat. Models that have Fear themselves are immune to the effects of Fear.

\subsubsection{Fearless}
\idx{Flee Charge Reaction}\idx[main=y]{\fearless}\label{fearless}

If more than half of a unit's models are Fearless, the unit automatically passes \hyperref[panic_test]{Panic Tests} and cannot declare a Flee Charge Reaction, unless already Fleeing. Models that are Fearless are also immune to the effects of \hyperref[fear]{Fear}.

\subsubsection{Feigned Flight}
\idx[main=y]{\feignedflight}\label{feigned_flight}

If a unit consisting entirely of models with Feigned Flight voluntarily chooses Flee as Charge Reaction and passes its \hyperref[rally_fleeing_units]{Rally Test} in its next Player Turn, it does not become \hyperref[shaken]{Shaken}. The \hyperref[reform]{Reform} after Rallying in this case does not prevent the unit from moving nor from shooting, but the unit still counts as having moved. This rule does not apply if the unit Flees involuntarily (e.g. as the result of a failed \hyperref[panic_test]{Panic Test}, or if it was already Fleeing when being Charged).

\subsubsection{Fly (X, Y)}
\idx[main=y]{Flying Movement}\idx[main=y]{\fly{}{}}\label{fly}

Units composed entirely of models with Fly may use Flying Movement during \hyperref[charge_move]{Charge Moves}, \hyperref[failed_charge]{Failed Charge Moves}, \hyperref[advance_move]{Advance Moves}, and \hyperref[march_move]{March Moves}. When a unit uses Flying Movement, substitute its models' Advance Rate with the first value given in brackets (X), and their March Rate with the second value given in brackets (Y). A unit using Flying Movement ignores, and is ignored by all Terrain Features and units during the Flying Movement, except for the Charged unit during a Charge Move. Note that:

\begin{itemize}
	\item It must follow the \hyperref[unit_spacing]{Unit Spacing} rule at the end of the move.
	\item It is affected by the Terrain Features from which it takes off and in which it lands.
	\item All modifiers to ground movement values also apply to a model's Fly values, unless specifically stated otherwise.
	\item When declaring a Charge with a unit with Fly, you must declare if the unit will not use Flying Movement for the Charge Move.
	\item A Failed Charge Move of a unit with Fly must use the type of movement (ground or Flying) that was chosen when the Charge was declared. If the unit would end its Failed Charge Move inside another unit's Unit Boundary or inside Impassable Terrain, backtrack the move to the unit's last legal position where it follows the \hyperref[unit_spacing]{Unit Spacing} rule.
\end{itemize}

\subsubsection{Frenzy}
\idx[main=y]{\frenzy}\label{frenzy}

At the start of the Charge Phase, each of your non-Fleeing units with at least one model with Frenzy that is unengaged, does not contain any \hyperref[shaken]{Shaken} models, and has an enemy unit inside its Front Arc within the unit's Advance Rate +\distance{7} must take a \hyperref[discipline_tests]{Discipline Test}, called a Frenzy Test. If the test is failed, the whole unit must declare a Charge this Player Turn if possible.

Units with at least one model with Frenzy gain \textbf{Maximised (Frenzy Tests, Restrain Pursuit Tests)}.


If there are different Advance Rates available in the unit, the Advance Rate used for the Frenzy Test and for the Charge Range is determined as follows:

\begin{itemize}
	\item If a model has more than one Advance Rate (e.g. due to \hyperref[fly]{Fly}), the model must use the Advance Rate that has the highest chance of completing the Charge.
	\item If a unit contains models with different Advance Rates, the unit must use the highest Advance Rate that all models in the unit can use (which will usually be the lowest Advance Rate in the unit).
\end{itemize}

For example, a model with Advance Rate \distance{2} and Fly (\distance{8}, \distance{16}) must use the Advance Rate from Fly. And if a Character in a Combined Unit has Advance Rate \distance{4} while the \rnf{} models have \distance{6}, the Combined Unit must use Advance Rate \distance{4}. Note that when a unit is forced to declare a Charge due to a failed Frenzy Test, it is not forced to Charge the enemy unit that triggered the Frenzy Test.

\subsubsection{Front Rank}
\idx{Characters}\idx{Champions}\idx[main=y]{\frontrank}\label{front_rank}

Front Rank specifies where in a unit the model may be placed and how the model moves inside its unit.

Models with and models without Front Rank must be placed inside their units so that the following conditions are satisfied as best as possible, in decreasing priority order:

\begin{itemize}
	\item \nth{1} priority: The Front Facings of models without Front Rank must be placed as far backwards as possible.
	\item \nth{2} priority: The Front Facings of models with Front Rank must be placed as far forwards as possible.
\end{itemize}

\idx{Advance Move}\idx{March Move}When making an Advance Move, March Move, or Reform with a unit that includes models with Front Rank, these models can be reorganised into a new position (still as far forwards as possible) as part of the move. This counts towards the distance moved by the unit (measure the distance from the starting position to the ending position of the centre of the model with Front Rank to determine how far it has moved).

A model with Front Rank can either have a Matching Base or a Mismatching Base.

\paragraph{Matching Bases}
\idx[main=y]{Matching Bases}\label{matching_bases}

In Combined Units containing Characters and \rnf{} models, a Character is considered to have a Matching Base if:

\begin{itemize}
	\item The model has the same base size as the \rnf{} models.
	\item The model's base is the same size as a multiple of the \rnf{} models' bases (such as a \num{40}\timess{}\num{40} \si{\milli\meter} base in a \num{20}\timess{}\num{20} \si{\milli\meter} unit).
\end{itemize}

For Combined Units consisting entirely of Characters, Matching Bases are determined differently as these units do not contain any \rnf{} models. The \rnf{} base size for the purposes of Matching Bases must:
\begin{itemize}
	\item Correspond to the base size of at least one of the Characters.
	\item Result in as few Characters as possible having Mismatching Bases; the owner chooses in case of a tie.
\end{itemize}

For example, in a unit consisting of a 25\timess{}25 \si{\milli\meter} Character and two 25\timess{}50 \si{\milli\meter} Characters, that base size is 25\timess{}25 \si{\milli\meter} as it does not result in any Mismatching Bases in the unit.

If the first rank is occupied by models with Front Rank, a model with Matching Base is placed in the second rank instead. If this rank is also occupied by models with Front Rank, it is placed in the third rank, and so on. Matching Bases are subject to the following rules and restrictions:
\begin{itemize}
	\item If the model has a larger base than the \rnf{} models, it is considered to be in all ranks its base occupies for the purposes of calculating \hyperref[full_ranks]{Full Ranks}. For calculating the number of models in the unit's ranks (e.g. for Full Ranks, \hyperref[line_formation]{Line Formation}, \hyperref[area_attack]{Area Attack}), the large base counts as the number of models it displaces, or would displace if there aren't enough models.
	\item If a model with a Matching Base has a longer base than the \rnf{} models in the unit, the unit is allowed to have more than one incomplete rank if all incomplete ranks after the first consist entirely of models with such bases (for instance, the rear parts of long bases such as War Platforms are allowed to form several incomplete ranks).
	\item A model cannot join a unit that has more than one rank if its base is wider than the unit it wishes to join, nor can a unit \hyperref[reform]{Reform} into a formation that is narrower than any model joined to the unit.
\end{itemize}

If a model with Front Rank moves inside or leaves a unit that has more than one rank, or if it is removed from such a unit as a casualty,  the model's spot must be filled with models without Front Rank. If there aren't any models without Front Rank available, move models with Front Rank instead. Sometimes models with Front Rank must be redistributed in order for all such models to be as far forwards as possible. When this happens, immediately move as few models as possible in order to have all models with Front Rank as far forwards as possible.

If a model with Front Rank moves inside or leaves a unit that has a single rank, or if it is removed from such a unit as a casualty, gaps may be created in the unit. If this leads to an illegal formation (there can only be gaps in an incomplete rear rank; see \totalref{units}), follow the rules in \totalref{illegal_formation_after_removing_models}.

\columnbreak

\paragraph{Mismatching Bases}
\idx[main=y]{Mismatching Bases}\label{mismatching_bases}

Anything that is not a Matching Base is a Mismatching Base (such as a \num{50}\timess{}\num{75} \si{\milli\meter} base inside a \num{25}\timess{}\num{50} \si{\milli\meter} unit).

A model with Mismatching Base is always placed in base contact to the side of the unit, aligned with its front. Only two Mismatching Bases can be joined to a single unit (one at each side). These models are considered to be only in the first rank, but are ignored when counting the number of models in each rank in order to establish the number of \hyperref[full_ranks]{Full Ranks} and whether or not a unit is in \hyperref[line_formation]{Line Formation}. They form a file of one model each.

During Advance Moves, March Moves, and Reforms, models with Mismatching Bases can only be moved to the other side of the unit as part of the move.

Figure \ref{figure/front_rank} shows how models with Front Rank can be placed in a complex case.

\newcommand{\figFRA}{a)}
\newcommand{\figFRB}{b)}

\begin{Figure}
	\Fanchor
	\centering
	\def\svgwidth{\columnwidth}
	\subimport{../pics/}{front_rank.pdf_tex}
	\caption{Illustration of the \frontrank{} rule.\captionposttitle
		Yellow models have Front Rank, green models do not.\captionpar
		a) A Character on a Mismatching Base is placed next to the unit. Characters on Matching Bases are placed inside the unit, as far forwards as possible. This unit is considered to have 3 Full Ranks.\captionpar
		b) When a model with Front Rank joins the unit, the small model with Front Rank in the second rank must be moved to the side in order to have all models with Front Rank as far forwards as possible.
	}
	\label{figure/front_rank}
\end{Figure}

\subsubsection{Ghost Step}
\idx[main=y]{\ghoststep}\label{ghost_step}

The model may choose to treat all Terrain Features as \hyperref[open_terrain]{Open Terrain} for movement purposes, but must follow the \hyperref[unit_spacing]{Unit Spacing} rule upon the completion of its moves. It can never end its move inside Impassable Terrain. If this would be the case, backtrack the move to the unit's last legal position (unless Fleeing, in which case the normal rules for \totalref{flee_moves} apply).

In addition, the model automatically passes Dangerous Terrain Tests taken due to Terrain.

\columnbreak

\subsubsection{Hidden}
\idx[main=y]{\hidden}\label{hidden}

The model cannot be deployed during the Deployment Phase but \textbf{must} be deployed using Special Deployment rules.  At the start of any Player Turn, the owner may choose a friendly unit that meets all the following conditions:
\begin{itemize}
	\item The model with Hidden could have  been deployed in it if it weren't for Hidden and any other Special Deployment.
	\item It has the same Type and Height as the model with Hidden.
	\item It contains 2 or more \rnf{} models with the same base size as the model with Hidden, at least one of which must be in the first rank.
	\item It is not Fleeing nor a Summoned unit.
\end{itemize}
Then apply the following procedure:
\begin{itemize}
	\item Remove a \rnf{} model with the same base size as the model with Hidden from the chosen unit’s first rank as a casualty. Removing the model never causes a Panic Test.
	\item Deploy the model with Hidden in the position of the removed model.
	\item The model with Hidden cannot voluntarily leave its unit during the Player Turn in which it was deployed.
	\item If the model is not deployed by the end of Game Turn 4, it counts as a casualty and cannot be deployed for the rest of the game.
\end{itemize}
The model with Hidden is subject to the same ongoing effects as its unit, and counts as Charging if its unit Charged.

\subsubsection{Hold the Line}
\idx[main=y]{\holdtheline}\label{hold_the_line}

 Friendly units within \distance{6} of a non-Fleeing unit containing one or more models with Hold the Line \textbf{must} reroll failed Panic Tests.

\subsubsection{Insignificant}
\idx[main=y]{\insignificant}\label{insignificant}

Units consisting entirely of Insignificant models do not cause \hyperref[panic_test]{Panic Tests} on friendly units in which less than half the models  are Insignificant. Units with Insignificant \rnf{} models can only be joined by Insignificant Characters.

\subsubsection{Light Troops}
\idx{Full Ranks}\idx[main=y]{\lighttroops}\label{light_troops}

A unit composed entirely of models with Light Troops applies the following rules for Advance Moves and March Moves:

\begin{itemize}
	\item The unit may perform any number of \hyperref[reform]{Reforms}, at any time during the move, and in any order. This does not prevent models with Light Troops from shooting this Player Turn.
	\item The unit can move backwards and sideways as if moving forwards (i.e. up to its Advance/March Rate, and a unit can combine backwards, sideways, and forwards movement), but cannot leave the board with any part of its Unit Boundary.
	\item The unit cannot perform any Wheels.
\end{itemize}

In addition:
\begin{itemize}
	\item Units composed entirely of models with Light Troops gain \hyperref[march_and_shoot]{\textbf{March and Shoot}}.
	\item Units with more than half of their models with Light Troops always count as having 0 \hyperref[full_ranks]{Full Ranks}.
	\item Infantry Characters gain Light Troops while joined to Infantry units of the same Height with Light Troops, and lose this instance of Light Troops when leaving that unit.
\end{itemize}

\subsubsection{Magic Resistance (X)}
\idx[main=y]{\magicresistance{}}\label{magic_resistance}

Learned Spells and Bound Spells targeting at least one enemy unit, including a model or model part inside a unit, with one or more models with Magic Resistance suffer a \minuss{}X modifier to their casting roll (where X is given in brackets). If there are different X values that could be used, use the highest value.

\subsubsection{Massive Bulk}
\idx[main=y]{\massivebulk}\label{massive_bulk}

If the model is mounted by a Character, ignore the rider's Armour Equipment (including Armour Enchantments) and Personal Protections, unless specifically stated otherwise (such as Armour Enchantments that affect the bearer's model).

\subsubsection{Maximised (X)}
\idx[main=y]{Maximised}\label{maximised}

The model's dice rolls stated in brackets are subject to \hyperref[maximised_roll]{Maximised Roll}. This rule is cumulative.

\subsubsection{Minimised (X)}
\idx[main=y]{Minimised}\label{minimised}

The model's dice rolls stated in brackets are subject to \hyperref[maximised_roll]{Minimised Roll}. This rule is cumulative.

\subsubsection{Not a Leader}
\idx{General}\idx[main=y]{\notaleader}\label{not_a_leader}

The model cannot be the \hyperref[the_general]{General}.

\subsubsection{Protean Magic}
\idx[main=y]{\proteanmagic}\label{protean_magic}

The Wizard does not choose an available Path of Magic to select spells from. Instead, during \hyperref[spell_selection]{Spell Selection}, the Wizard must select its spells from the \hyperref[learned_spells]{Learned Spell} 1 of each Path it has access to, as well as the \hyperref[hereditary_spells]{Hereditary Spell} of its army. This rule overrides the Spell Selection rules for Wizard \hyperref[wizard_apprentice]{Apprentices}, \hyperref[wizard_adept]{Adepts}, and \hyperref[wizard_master]{Masters}.

\subsubsection{Rally Around the Flag}
\idx[main=y]{\rallyaroundtheflag}\label{rally_around_the_flag}

All models in units, including Fleeing units, within \distance{12} of a friendly non-Fleeing model with Rally Around the Flag may reroll failed \hyperref[discipline_tests]{Discipline Tests}.

\columnbreak

\subsubsection{Random Movement (X)}
\idx[main=y]{\randommovement{}}\label{random_movement}

At the end of step 2 of the \hyperref[the_movement_phase_sequence]{Movement Phase Sequence} (after Rallying Fleeing Units), a non-Fleeing unit with Random Movement must perform a Pursuit Move, with the following exceptions that only apply in the Movement Phase, unless specifically stated otherwise:

\begin{itemize}
	\item It \textbf{always} moves the distance stated in brackets (X), which is also used for Flee Distance and Pursuit Distance (including \hyperref[overrun]{Overruns}).
	\item It \textbf{must} choose which direction it will move in before Pivoting and rolling the Pursuit Distance.
	\item It cannot move off the Board Edge using the rules for Pursuing off the Board (it can still overlap the Board Edge with parts of its Unit Boundary other than its Front Facing).
	\item It does not take \hyperref[dangerous_terrain]{Dangerous Terrain Tests} unless Charging.
\end{itemize}


There are several restrictions connected with Random Movement:

\begin{itemize}
	\item The unit cannot move normally in the Movement Phase (Advance, March, Reform) and cannot declare Charges in the Charge Phase. Whenever it requires a March Rate (e.g. when \hyperref[post_combat_reform]{Post-Combat Reforming}), use the potential maximum value of X as its March Rate.
	\item The unit cannot perform \hyperref[magical_move]{Magical Moves}.
	\item The unit cannot use \hyperref[swiftstride]{Swiftstride} (but X can be affected by Maximised/Minimised Roll from other sources).
	\item Characters with Random Movement cannot join units, and units with Random Movement cannot be joined by Characters. Note that Characters that are part of a Combined Unit when the unit gains Random Movement will gain Random Movement too as they are already part of that unit.
	\item If the unit has several instances of Random Movement, use the one with the lowest average (the owner chooses in case of a tie).
\end{itemize}

\columnbreak

\subsubsection{Scoring}
\idx[main=y]{Scoring Units}\idx[main=y]{\scoring}\label{scoring}

Units with at least one model with Scoring are considered to be Scoring Units, which are used for winning Secondary Objectives (see \totalref{secondary_objectives}). Every army needs Scoring Units to be able to complete Secondary Objectives, which is why units with Scoring are marked in the Army Books with a special pennant icon:

\begin{center}
	\textcolor{white}{debug}\includegraphics[width=2.5cm]{../Layout/pics/logo_scoring.png}\textcolor{white}{debug}
\end{center}

Scoring can be lost during the game:
\begin{itemize}
	\item A unit that is Fleeing loses Scoring for as long as it is Fleeing.
	\item An \hyperref[ambush]{Ambushing} unit that enters the Battlefield on Game Turn 4 or later loses Scoring for the rest of the game.
	\item A unit that has performed a \hyperref[post_combat_reform]{Post-Combat Reform} loses Scoring until the start of the following Player Turn.
	\item A \hyperref[vanguard]{Vanguarding} model loses Scoring until the end of Game Turn 1.
\end{itemize}

\subsubsection{Scout}
\idx{Special Deployment}\idx[main=y]{\scout}\label{scout}

Units with Scout may be deployed using Special Deployment rules. All units that will be deployed using the Scout rule must be declared at step 8 of the Pre-Game Sequence (after Spell Selection), starting with the player who chose their Deployment Zone. Scout deployment is conducted on Step 5 of the Deployment Phase (Deploy Scouting Units). If both players have Scouting units, alternate unit placement (one unit at a time), starting with the player who first completed their normal deployment. Scouting units have three deployment options:
\begin{itemize}
	\item Fully inside your Deployment Zone, using the normal deployment rules
	\item Anywhere on the Battlefield at least \distance{18} away from enemy units
	\item Anywhere on the Battlefield fully inside a Field, Forest, Ruins, or Water Terrain Feature and at least \distance{12} away from enemy units
\end{itemize}
Scouting units that aren't placed fully inside their Deployment Zone may not declare Charges in the first Player Turn of the first Game Turn (there are no Scout Charge restrictions after the first Player Turn).

\columnbreak

\newcommand{\figSkirmiA}{a)}
\newcommand{\figSkirmiB}{b)}
\newcommand{\figSkirmiDist}{\normalfontsize \SI{12.5}{\milli\meter}}
\newcommand{\figSkirmiCharOne}{$C_{1}$}
\newcommand{\figSkirmiCharTwo}{$C_{2}$}

\begin{figure*}[!htbp]
	\renewcommand{\figbiglettersize}{20}
	\centering
	\def\svgwidth{\textwidth}
	\subimport{../pics/}{skirmisher.pdf_tex}
	\caption{Skirmish formation.\captionposttitlepremc%
	\raggedcolumns\begin{multicols}{2}%
		a) An example of a unit in skirmish formation with a joined Mismatching Character.\columnbreak\newline
		b) The same unit Engaged in Combat. Models with bold frame can attack a Character (either $C_{1}$ or $C_{2}$).  Models with dashed frame cannot attack at all.%
	\end{multicols}%
	}
	\label{figure/skirmisher}
	\vspace*{-20pt}
\end{figure*}

\subsubsection{Skirmisher}
\idx[main=y]{\skirmisher}\label{skirmisher}

The model can always use Shooting Attacks from any rank (models with Skirmisher are not limited to shooting from first and second rank).

Units with at least one \rnf{} model with Skirmisher are formed into a skirmish formation. They are not placed in base contact with each other. Instead, models are placed with a \SI{12.5}{\milli\meter} distance (roughly half an inch) between them. This gap is considered part of the unit for Cover purposes, and will have the same Height as the models in the unit. Other than this gap between models, units with Skirmisher follow the normal rules for forming units and therefore have a Front, two Flank, and a Rear Facing, can perform Supporting Attacks, and so on. Units in skirmish formation never block Line of Sight (remember that this also affects Cover as they can never contribute to Hard Cover).

Units in skirmish formation can only be joined by Characters that have both the same Type and the same Height as the unit. Unless a Character has the exact same base size as all \rnf{} models in the unit, it is considered Mismatched for the purpose of placement within the unit. The unit ceases to be in skirmish formation when all \rnf{} models with Skirmisher are removed as casualties: immediately contract their skirmish formation into a normal formation, without moving the centre of the first rank. Nudge any unit as normal to maintain base contact if possible.

See figure \ref{figure/skirmisher} for an illustration of this rule.

\subsubsection{Special Ambush (X)}
\idx[main=y]{\specialambush}\label{special_ambush}

The model follows the rules for Ambush with the following exception: when its unit enters the Battlefield, apply the following rules instead of placing it using the normal Ambush rules:
\begin{enumerate}
	\item Choose a point on the Battlefield that meets the conditions specified in brackets (X) and place the unit in contact with and fully within \distance{6} of the chosen point. The unit can be placed in any formation but must follow the Unit Spacing rule.
	\item If there is no point available that enables the unit  to be placed according to these rules,  the unit cannot enter the Battlefield during this Player Turn. Roll again in the next friendly Player Turn.
	\item If deployed in the Movement Phase, the unit is Shaken until the end of the phase .
	\item For the purpose of shooting, the unit counts as having performed a March Move in this Player Turn.
\end{enumerate}

\subsubsection{Stand Behind}
\idx[main=y]{\standbehind}\label{stand_behind}

The model can be placed anywhere inside its unit (its Front Facing doesn't have to be placed as far forwards as possible and it can be placed farther backwards than that of models without \hyperref[front_rank]{Front Rank}, even if the model has \hyperref[front_rank]{Front Rank}). Its Front Facing cannot be placed farther forwards inside a unit than that of any model with Front Rank but without Stand Behind. Ignore Stand Behind for models with Mismatching Bases.

\subsubsection{Strider}
\idx[main=y]{\strider{}}\label{strider}

The model automatically passes \hyperref[dangerous_terrain]{Dangerous Terrain} Tests caused by Terrain. If more than half of a unit's models have Strider, the unit never loses \hyperref[steadfast]{Steadfast} due to Terrain. Sometimes Strider is linked to a specific type of Terrain, stated in brackets. In this case, Strider only applies when interacting with this type of Terrain.

\subsubsection{Stubborn}
\idx{Combat Reforms}\idx{Break Test}\idx[main=y]{\stubborn}\label{stubborn}

A unit with at least one model with Stubborn ignores Discipline modifiers from the Combat Score difference when taking \hyperref[break_test]{Break Tests} or \hyperref[combat_reform]{Combat Reform} Discipline Tests.

\subsubsection{Supernal}
\idx[main=y]{\supernal}\label{supernal}

All attacks made by the model become \hyperref[magical_attacks]{\textbf{Magical Attacks}}, including Special Attacks and \hyperref[crush_attack]{Crush Attacks}. In addition, the model gains \hyperref[unstable]{\textbf{Unstable}}, with the following exception: when a unit consisting entirely of models with Supernal loses a combat, it must take a \hyperref[break_test]{Break Test} (\hyperref[stubborn]{Stubborn} or \hyperref[steadfast]{Steadfast} units ignore Discipline modifiers from the Combat Score difference as normal):
\begin{itemize}
	\item If the Break Test is passed, ignore all Health Points that would be lost due to Unstable.
	\item If the Break Test is failed, follow the rules for Unstable as normal.
\end{itemize}

\subsubsection{Swift Reform}
\idx[main=y]{Swift Reform}\label{swift_reform}

During the Movement Phase, a unit containing one or more models with Swift Reform may execute a Swift Reform instead of a \hyperref[reform]{Reform}. The unit makes a Reform with the following exceptions:
\begin{itemize}
	\item The unit is not prohibited from shooting in the next Shooting Phase (but will still suffer the to-hit modifier for Moving and Shooting).
	\item The unit can perform an \hyperref[advance_move]{Advance Move} after the Reform. For the purpose of no model being able to end its movement with its centre farther away than its Advance Rate from its starting position, measure this distance after the Reform.
	\item No model can end its movement (after an Advance Move) with its centre farther away than its March Rate from its starting position before the Reform.
\end{itemize}

\subsubsection{Swiftstride}
\idx[main=y]{\swiftstride}\label{swiftstride}

If a unit is composed entirely of models with Swiftstride, it gains \textbf{Maximised (Charge Range, Flee Distance, Pursuit Distance, Overrun Distance)}.

\subsubsection{Tall}
\idx[main=y]{\tall}\label{tall}

\hyperref[line_of_sight]{Line of Sight} drawn to or from a model or \hyperref[boundary_rectangle]{Unit Boundary} with Tall is not blocked by models of the same Height (as the model or Unit Boundary with Tall), unless the intervening model also has Tall. Remember that this also affects \hyperref[cover]{Cover} (if a model blocks Line of Sight, it contributes to Hard Cover, otherwise only to Soft Cover). The Unit Boundary of units with half their models or more with Tall is considered to be Tall for the purpose of drawing Line of Sight to the Unit Boundary and determining if its unit benefits from Cover.

\subsubsection{Terror}
\idx[main=y]{\terror}\label{terror}

Units with more than half of their models with Terror are immune to the effects of Terror. When a unit with one or more models with Terror declares a Charge, its target must immediately take a \hyperref[panic_test]{Panic Test} before declaring its Charge Reaction. If the test is failed, the target of the Charge must declare a Flee Charge Reaction if able to do so.

\subsubsection{Towering Presence}
\idx[main=y]{\toweringpresence}\label{towering_presence}

The model gains \hyperref[tall]{\textbf{Tall}}, \hyperref[commanding_presence]{Commanding Presence (\distance{+6})}, \hyperref[rally_around_the_flag]{Rally Around the Flag (\distance{+6})} and, unless it is a \hyperref[war_platform]{War Platform}, \textbf{\exclusive{}}.  The \hyperref[boundary_rectangle]{Unit Boundary} of units with half their models or more with Towering Presence is considered to have Towering Presence for the purpose of drawing \hyperref[line_of_sight]{Line of Sight} to the Unit Boundary and determining if its unit benefits from \hyperref[cover]{Cover}.

\subsubsection{Unbreakable}
\idx[main=y]{\unbreakable}\label{unbreakable}

The model gains \textbf{Exclusive (Unbreakable)}, and the model’s unit automatically passes all \hyperref[break_test]{Break Tests}.

\subsubsection{Undead}
\idx[main=y]{\undead}\label{undead}

The model gains \hyperref[unstable]{\textbf{Unstable}}. Models with Undead cannot perform March Moves, unless their unit starts the March Move within the range of a friendly model's \hyperref[commanding_presence]{Commanding Presence}. The only Charge Reaction a unit with one or more models with Undead can perform is Hold.

When units consisting entirely of models with Undead lose Health Points due to \hyperref[unstable]{Unstable}, the number of lost Health Points can be reduced in certain situations. Apply the modifiers in the following order:
\begin{enumerate}
	\item If the unit contains at least one model with \hyperref[stubborn]{Stubborn}, halve the number of lost Health Points, rounding fractions up.
	\item If the unit is \hyperref[steadfast]{Steadfast}, ignore any excess Health Point losses above 12.
\end{enumerate}

\begin{enumerate}[resume]
	\item If the unit receives \hyperref[rally_around_the_flag]{Rally Around the Flag}, reduce the number of lost Health Points by the unit's current Rank Bonus. Units without any Rank Bonus reduce the number of lost Health Points by 1 instead.
	\item Apply all other modifiers (from Special Items, Model Rules, spells, etc.) afterwards.
\end{enumerate}

\columnbreak

\subsubsection{Unstable}
\idx[main=y]{\unstable}\label{unstable}

The model gains \textbf{Exclusive (Unstable)}. A unit with one or more models with Unstable does not take a \hyperref[break_test]{Break Test} when losing a Round of Combat, but instead it loses one Health Point for each point of Combat Score difference by which it lost the Round of Combat (with no saves of any kind allowed).

The Health Point losses are allotted in the following order:
\begin{enumerate}
	\item \rnf{} models, excluding Champions
	\item Champion
	\item Characters, allotted by the owner of the unit as evenly as possible
\end{enumerate}

\subsubsection{Vanguard}
\idx{Special Deployment}\idx[main=y]{\vanguard}\label{vanguard}

After Deployment (including \hyperref[scout]{Scouting} units), models with Vanguard may perform a \distance{12} move. This move is performed as a combination of \hyperref[advance_move]{Advance Move} and/or \hyperref[reform]{Reforms}, as in the Movement Phase, including any actions and restrictions that normally apply to the unit (e.g. \hyperref[pivots_and_wheels]{Wheeling}, joining units, leaving units, and so on). The \distance{12} distance is used instead of the unit's Advance Rate and March Rate. In case a figure is stated in brackets, this distance is \distance{X} instead.

This move cannot be used to move within \distance{12} of enemy units. This is decreased to \distance{6} for enemy units that have either Scouted or Vanguarded.

If both players have units with Vanguard, alternate moving units one at a time, starting with the player who completed their normal deployment last (note that this is an exception to the rules for Simultaneous Effects). A Combined Unit counts as a single unit for this purpose, even if parts of the unit perform separate Vanguard moves (like two Characters Vanguarding out of a Combined Unit). Any game effects that would affect the Combined Unit (such as Banner Enchantments) remain in effect for all parts of the Combined Unit until all parts of the Combined Unit have finished their Vanguard move (even if a Character leaves the unit). Instead of moving a unit, a player may declare to not move any more Vanguarding units.

Units that have moved this way lose Scoring until the end of Game Turn 1 and may not declare Charges in the first Player Turn (if their side has the first turn).

\subsubsection{War Machine}
\idx[main=y]{\warmachine}\label{war_machine}

The model gains \textbf{\exclusive{}} and cannot Pursue (which does not prevent it from being affected by Random Movement), declare Charges, or declare Flee Charge Reactions.

When a War Machine fails a \hyperref[panic_test]{Panic Test}, instead of Fleeing, it is \hyperref[shaken]{Shaken} until the end of the next Player Turn. War Machines that fail a \hyperref[break_test]{Break Test} are automatically destroyed and removed as casualties. War Machines on round bases and units Engaged in Combat with them cannot make \hyperref[combat_reform]{Combat Reforms}.

When a unit Charges a War Machine on a round base, it can move into base contact by having its Front Facing contact any point of the War Machine's base (it must still maximise the number of models in base contact, see \totalref{contact_between_objects} and figure \ref{figure/empty_gaps}, page \pageref{figure/empty_gaps}). No \hyperref[aligning_units]{align move}\idx{Aligning Units} is allowed.

When a unit Breaks from Combat and Flees away from a War Machine on a round base, always Pivot the Fleeing unit \SI{180}{\degree}, so that it's Rear Facing is in contact with the War Machine's base. Otherwise follow the normal rules for units Breaking from Combat and Fleeing.

\subsubsection{War Platform}
\idx[main=y]{\warplatform}\label{war_platform}

Unless selected as a mount for a Character, a model with War Platform gains \hyperref[characters]{\textbf{Character}} with the following exceptions:

\begin{itemize}
	\item It does not count towards the Characters Army Category (for Army List creation).
	\item It does not count as Character when Deploying Units (it may still be deployed inside units).
	\item It cannot \hyperref[issuing_a_duel]{issue Duels}\idx{Issuing a Duel}\idx{Duels}, \hyperref[accepting_and_refusing_a_duel]{accept Duels}, or \hyperref[make_way]{Make Way}.
	\item It can perform \hyperref[swirling_melee]{Swirling Melee}.
	\item It does not count as Character regarding \hyperref[bodyguard]{Bodyguard} and \hyperref[multiple_wounds]{Multiple Wounds}, unless the War Platform is specifically mentioned in the Bodyguard rule.
\end{itemize}

The model can join units , and having \exclusive{} (\hyperref[chariot]{Chariot}) does not prevent it from joining units without Chariot. Additionally, it does not prevent Characters without \exclusive{} (\hyperref[chariot]{Chariot}) from joining a unit containing a model with War Platform and Chariot. When joined to a unit, it must always be placed in the centre of the first rank, possibly pushing back other models with \hyperref[front_rank]{Front Rank}, and must keep its position in the centre of the first rank at all times, including during casualty removal (as long as it is joined to the unit). Should the model ever be forced to leave its position in the centre of the first rank, immediately redistribute models in the unit until the model is back in the centre of the first rank, following the priority order in 16.B.e Illegal Formation after Removing Models. If two positions are equally central (e.g. in a unit with an even number of models in the first rank and a \hyperref[war_platform]{War Platform} replacing an uneven number of models per rank), the War Platform can be placed in either of these positions. If the War Platform cannot be placed in the centre of the the first rank, the model cannot join the unit. A War Platform with Mismatching Base can never join a unit, and only a single War Platform can be in the same unit unless specifically stated otherwise.

\subsubsection{Wizard Apprentice}
\idx[main=y]{\wizardapprentice}\label{wizard_apprentice}

The Wizard selects its spells as described in \totalref{spell_selection}.

\subsubsection{Wizard Adept}
\idx[main=y]{\wizardadept}\label{wizard_adept}

The Wizard gains \textbf{\hyperref[channel]{Channel (1)}} and selects its spells as described in \totalref{spell_selection}.

\subsubsection{Wizard Master}
\idx[main=y]{\wizardmaster}\label{wizard_master}

The Wizard gains \textbf{\hyperref[channel]{Channel (1)}} and a +1 modifier to its casting rolls, and selects its spells as described in \totalref{spell_selection}.

\subsubsection{Wizard Conclave}
\idx[main=y]{\wizardconclave}\label{wizard_conclave}

The Champion of a unit with Wizard Conclave is a \hyperref[wizard_adept]{\textbf{Wizard Adept}} and gains +1 Health Point in addition to the normal Attack Value increase associated with being a \hyperref[champion]{Champion}. This Champion may select up to two spells from predetermined spells given in the unit entry. This overrides the \hyperref[spell_selection]{Spell Selection} rules for Wizard Adepts.

\columnbreak

\section{Character}
\idx[main=y]{Characters}\label{characters}

Character is a special type of Universal Rule. Unless specifically stated otherwise, any model bought as part of the Characters Army Category of an Army Book has the Character Universal Rule. A model with this rule is referred to as a Character.

All Characters gain the \hyperref[front_rank]{\textbf{Front Rank}} Universal Rule.

\idx[main=y]{Lone Characters}\subsection{Lone Characters}

Characters can operate as a unit consisting of just a single model. In this case, follow the normal rules for units.

\subsection{Characters Joined to Units}

Characters can operate as part of other units by joining them. This can be done either by deploying the Character in the unit, or by moving into contact with the unit during the Movement Phase while performing an Advance Move or a March Move. If a Character being deployed in a unit would cause the Combined Unit to no longer be fully inside its Deployment Zone, the Character cannot be deployed in that unit. Units that are Engaged in Combat or Fleeing cannot be joined.

Characters can join other Characters to form a unit consisting only of Characters.

\idx[main=y]{Combined Units}Units that are formed by Characters joining \rnf{} models or other Characters are referred to as Combined Units.

When a Character joins a unit, it must move into a legal position during its Advance or March Move (see \totalref{front_rank}). A Character can choose any legal position it can reach with its move, moving through the unit it joins, possibly displacing other models (including models with Front Rank). Move displaced models as little as possible in order to keep all models in legal positions. If the Character does not have a sufficient movement range to reach a legal position, it cannot join the unit.

When a Character joins a unit with just a single rank, the owner can choose to either displace a model to the second rank, or to expand the unit's width and place the displaced model at either side of the first rank.

When a unit is joined by a Character, the unit cannot move any farther in the same Movement Phase. For determining which model counts as having moved or Marched (e.g. for purposes of shooting), the Character and the unit are treated individually during the Player Turn in which the Character joined the unit. For example, if the unit hasn't moved and the Character has Marched in order to join the unit, the Character counts as having Marched, while the rest of the unit counts as not having moved at all.

Once joined to a unit, the Character is considered as part of the unit for all rules purposes.

\subsection[\rnf{} Models in a Combined Unit Wiped out]{\rnf{} Models in a Combined Unit\\ Wiped out}

If a Combined Unit has all its \rnf{} models removed as casualties, leaving one or more Characters behind, the remaining Characters will stay a Combined Unit, which is considered to be the same unit for ongoing effects (such as \hyperref[one_turn]{\oneturn} spells) and \hyperref[panic_test]{Panic} (the unit has not been destroyed; the Characters in this Combined Unit may have to take a Panic Test if they have suffered \SI{25}{\percent} or more casualties). They are treated as a new unit for \hyperref[rally_fleeing_units]{Rally Tests} (i.e. Characters that were part of Fleeing Combined Units at \SI{25}{\percent} or less of their starting number of Health Points take Rally Tests on their normal Discipline). Models with \hyperref[front_rank]{Front Rank} that are positioned in a rank other than the first when the last \rnf{} model is removed as a casualty remain in the rank they occupy (this may result in the reduction of files compared to the unit's original formation), unless this would cause an illegal formation (see \totalref{illegal_formation_after_removing_models}).

\subsection{Leaving a Combined Unit}

A Character can leave a Combined Unit in the Charge Phase and in the Movement Phase if it is able to move (i.e. if it isn't Engaged in Combat, hasn't already moved, isn't Fleeing, etc.). In both cases, any game effects that would affect the Combined Unit (such as Banner Enchantments) remain in effect for all parts of the Combined Unit until the end of the phase (even if a Character leaves the unit), unless specifically stated otherwise (e.g. \oneturn{} spells). The Character ignores models from the Combined Unit for movement purposes and may make a Flying Movement (if it has \hyperref[fly]{Fly}). Characters leaving a unit do not affect whether or not this unit counts as having moved (e.g. for purposes of shooting).

\subsubsection{Charging out of a Unit}

Declare a Charge with a Character in a Combined Unit (during the Charge Phase as normal) and apply the following rules:

\begin{itemize}
	\item Use the Character's model for determining Line of Sight and distance to the enemy unit.
	\item As soon as the Character declares a Charge, it is considered a separate unit (i.e. it uses its own Advance Rate, all hits from Stand and Shoot Charge Reactions will hit the Character, in case of a Flee Charge Reaction, the enemy unit Flees away from the Character, etc.).
	\item Ignore the unit the Character was part of when determining Line of Sight and cover for Stand and Shoot Charge Reactions.
	\item The unit itself (including other Characters in the unit) cannot declare Charges in the same Player Turn.
	\item If the Charge is successful, move the Character out of the unit by Charging as normal.
	\item If the Charge is not successful, the Character makes a \hyperref[failed_charge]{Failed Charge Move} out of the unit. If the Failed Charge Move is too short to place the Character outside \distance{1} of the Combined Unit, the Character is no longer considered a separate unit and remains in the Combined Unit, and all models in the Combined Unit are \hyperref[shaken]{Shaken} until the end of the Player Turn.
\end{itemize}

\subsubsection[Advance/March Moving out of a Combined Unit]{Advance/March Moving out of a\\ Combined Unit}

A Character counts as part of the unit until it has physically left it. If a Character does not have enough movement to be placed at least \distance{1} away from the unit, it cannot leave the unit. A Character cannot leave a unit and rejoin it in the same phase. If one or more Characters want to leave a Combined Unit during the Movement Phase, apply the following rules:

\begin{itemize}
	\item Choose the Combined Unit and take a single March Test if necessary as per \totalref{moving_units}. The test applies to both Characters and \rnf{} models in the unit, i.e. if the test is failed, none of the models may perform a March Move during this Movement Phase.
	\item Characters leaving the unit and the remainder of the Combined Unit can perform different types of move (see \totalref{moving_units}).
	\item Move any Character that can and wishes to leave the unit, then move the remainder of the unit if applicable.
	\item Once all elements of the chosen Combined Unit that can and wish to have moved, proceed with the next unit.
\end{itemize}

\subsection{Distributing Hits onto Combined Units}
\idx{Distributing Hits}

When a non-Close Combat Attack hits a Combined Unit, there are two possibilities for distributing hits:

\begin{itemize}
	\item If Characters are of the same Type \textbf{and the} same Height \textbf{and} there are 5 or more \rnf{} models in the unit, then all hits are distributed onto the \rnf{} \hyperref[health_pools]{Health Pool}, and Characters cannot suffer any hits, unless specifically stated otherwise.
	\item If Characters are of a different Type \textbf{or a} different Height \textbf{or} there are 4 or less \rnf{} models in the unit, then the player making the attack distributes hits onto the \rnf{} \hyperref[health_pools]{Health Pool} and Characters. In case of attacks that are not made by either player, the owner of the affected unit distributes the hits. All simultaneous hits must be distributed as equally as possible, meaning that no model can take a second hit until all models have taken a single hit, and so on.
\end{itemize}

If a unit of 5 or more \rnf{} models contains several Characters of both the same and different Type or Height, Characters with the same Type and Height as the \rnf{} models are ignored for the hit distributions. Note that hits are never distributed onto Champions, unless specifically stated otherwise.

\subsection{Make Way}
\idx[main=y]{Make Way}\label{make_way}

At step 3 of the \hyperref[round_of_combat_sequence]{Round of Combat Sequence}, any Character placed in the first rank and not in base contact with an enemy model may move into contact with an enemy model. This enemy model must be in base contact with the Character's unit, and it must be attacking the Character's unit in its Front Facing. To perform a Make Way move, the Character switches position with another model (or models) in its unit; these models cannot be Characters. Characters with Mismatching Bases can never perform a Make Way move.

\section{Command Group}
\idx[main=y]{Command Group}\label{command_group}

Certain units feature the option of upgrading regular models to a Champion, Musician, or Standard Bearer. If so, the model gains the corresponding Model Rule. These models are referred to as a unit's Command Group.

\subsection{Champion}
\idx[main=y]{Champions}\label{champion}

A Champion gains \textbf{First Amongst Equals}, \hyperref[front_rank]{\textbf{Front Rank}}, and \textbf{Ordering the Charge}.

\subsubsection{First Amongst Equals}
\idx[main=y]{First Amongst Equals}\label{first_among_equals}

A Champion gains +1 Attack Value. If it is a Multipart Model, the Attack Value modifier only affects a single model part, which must be a model part without \hyperref[harnessed]{Harnessed} or \hyperref[inanimate]{Inanimate}.

\subsubsection{Ordering the Charge}
\idx[main=y]{Ordering the Charge}\label{ordering_the_charge}

When a unit with a Champion rolls its Charge Range, it \textbf{always} counts as rolling at least a \result{4}. If the Charge is still failed, ignore this rule and use the rolled dice to determine the Failed Charge Move following the normal rules.

For example, a Charging unit with an Advance Rate of \distance{7} containing a Champion declares a Charge against an enemy unit that is \distance{11} away. In case of a Charge Range roll of 2, the Charge will still be successful since the Charge Range roll is considered to be 4, resulting in a Charge Range of \distance{11}.

\columnbreak

\subsubsection{Other Rules Associated with Champions}
\label{other_rules_associated_with_champions}

\begin{itemize}
	\item Hits from attacks that follow the rules for Distributing Hits are never distributed onto Champions (see page \pageref{distributing_hits}).
	\item When removing non-Champion \rnf{} models as casualties and a Champion is in a position that would normally be removed as a casualty, remove the next eligible \rnf{} model and slide the Champion into the empty spot if applicable (see \totalref{removing_casualties}).
	\item When Raising Health Points, a Champion is the first model that is brought back if it was previously removed as a casualty (see \totalref{raise_health_points}).
	\item Champions may choose to use a different Shooting Weapon than the other \rnf{} models in their unit (see \totalref{shooting_with_a_unit}).
	\item Champions may issue and accept Duels. If a Duel is not accepted, a Champion cannot be chosen as the model that suffers the penalties for refusing a Duel (see \totalref{duels}).
\end{itemize}

\subsection{Musician}
\idx[main=y]{Musicians}\label{musician}

A Musician gains \hyperref[swift_reform]{\textbf{Swift Reform}} and \textbf{March to the Beat}.

\subsubsection{March to the Beat}
\idx[main=y]{March to the Beat}\label{march_to_the_beat}

A unit within \distance{8} of one or more enemy units that contain a model with March to the Beat suffers \minuss{}1 Discipline when taking March Tests. Units with at least one model with March to the Beat ignore this modifier.

\subsection{Standard Bearer}
\idx[main=y]{Standard Bearers}\label{standard_bearer}

A Standard Bearer gains \textbf{Combat Bonus}. Certain Standard Bearers may have the option to be upgraded with Banner Enchantments (see \totalref{banner_enchantments}).

\subsubsection{Combat Bonus}
\label{combat_bonus_CG}

A side with Standard Bearers adds +1 to its Combat Score for each Standard Bearer.

\columnbreak

\subsection[Placing and Moving Command Group Models]{Placing and Moving Command\\ Group Models}
\label{moving_command_group_models}

Musicians and Standard Bearers can be placed anywhere inside their units.

When making an Advance Move, March Move, or Reform with a unit that includes a Musician and/or a Standard Bearer, these models can be reorganised into a new position as part of the move. This counts towards the distance moved by the unit (measure the distance from the starting position to the ending position of the centre of the Command Group model to determine how far it moved).

Note that Champions are placed and moved inside their units according to the rules in \hyperref[front_rank]{Front Rank} as normal.

\subsection[Removing Command Group Casualties]{Removing Command Group\\ Casualties}
\label{removing_command_group_casualties}

If a Musician or Standard Bearer is to be removed as a casualty, replace the closest non-Command Group \rnf{} model from the same Health Pool (if there is any) with the Musician or Standard Bearer. The owner chooses if several \rnf{} models are equally close. It is assumed that another soldier picked up their tool and responsibility. Champions are however not replaceable and have their own Health Pool, which can be specifically targeted in certain situations (e.g. by allocating Close Combat Attacks, attacks that target individual models such as attacks from \hyperref[focused]{Focused} spells, or attacks that target all models in a unit). When a Champion is removed as a casualty from a unit with more than one rank, usually a non-Champion \rnf{} model is moved to fill the empty spot, and in case of a unit with a single rank, other models may be slid to fill the empty spot. If enough Health Points are lost from a unit's Health Pool to remove all non-Champion \rnf{} models as casualties, any remaining Health Point losses are allotted to the Champion, even if it is fighting a Duel.

\section{Personal Protections}
\idx[main=y]{\personalprotections}\label{personal_protections}

If at least one model part has a Personal Protection, the entire Multipart Model follows the rules of the Personal Protection, unless the model's mount is of Gigantic Height (and therefore has the \hyperref[massive_bulk]{Massive Bulk} Universal Rule). In this case, only the mount's Personal Protections are applied.

For example, if a Character with Distracting mounts a horse (Standard Height), the Multipart Model is affected by Distracting. If the Character instead mounts a dragon (Gigantic Height), the Multipart Model is not affected by Distracting.

\subsection{Conditional Application}
\label{PP_conditional_application}

Personal Protections may only work against certain attacks, which are then stated in brackets after \enquote{against}. There may already be some piece of information relative to the rule specified between brackets, as in Aegis (4+). In this case, the conditions for the rule to work are written in the same brackets, after a comma. This can e.g. be certain kinds of attacks or attacks with a given Attack Attribute, like Aegis (4+, against Melee Attacks) or Aegis (2+, against Flaming Attacks).

\subsection{List of Personal Protections}
\label{list_of_personal_protections}

\subsubsection{Aegis (X)}
\idx[main=y]{\aegis{}}\label{aegis}

Aegis is a \hyperref[special_saves]{Special Save}. A model must reroll successful Aegis Saves against \hyperref[divine_attacks]{Divine Attacks}.

\subsubsection{Cannot be Stomped}
\idx[main=y]{\cannotbestomped}\label{cannot_be_stomped}

For the purposes of \hyperref[stomp_attacks]{Stomp Attacks} from enemy models, a model with Cannot be Stomped is never considered to be of Standard Height.

\subsubsection{Distracting}
\idx[main=y]{\distracting}\label{distracting}

Close Combat Attacks allocated towards a model with Distracting suffer a \minuss{}1 to-hit modifier. Ignore this rule if the attack is affected by any other negative to-hit modifier.

\subsubsection{Flammable}
\idx[main=y]{\flammable}\label{flammable}

\hyperref[flaming_attacks]{Flaming Attacks} must reroll failed to-wound rolls against a model with Flammable.

\subsubsection{Fortitude (X)}
\idx[main=y]{\fortitude{}}\label{fortitude}

Fortitude is a \hyperref[special_saves]{Special Save}. Fortitude Saves cannot be taken against attacks with \hyperref[lethal_strike]{Lethal Strike} that rolled a natural \result{6} to wound, or against \hyperref[flaming_attacks]{Flaming Attacks}.

\subsubsection{Hard Target (X)}
\idx[main=y]{\hardtarget}\label{hard_target}

Shooting Attacks targeting a unit that has more than half of its models with Hard Target (X) suffer a \minuss{}X to-hit modifier. This rule is cumulative.

\subsubsection{Immune (X)}
\idx[main=y]{\immune}\label{immune}

Attacks with the Attack Attributes stated in brackets (X) allocated towards or distributed onto the model lose these Attack Attributes.

\subsubsection{Parry}
\idx[main=y]{\parry}\label{parry}

Parry can only be used against \hyperref[melee_attacks]{Close Combat Attacks} from the Front Facing. The model gains one of the following effects, whichever would result in a higher Defensive Skill:
\begin{itemize}
	\item The model gains +1 Defensive Skill.
	\item The model's Defensive Skill is \textbf{always} set to the Offensive Skill of the attacker.
\end{itemize}

\section{Armour Equipment}
\idx{Armour (\ArmourInitials)}\label{armour_equipment}

There are different types of Armour Equipment. A model can only ever be equipped with one piece of armour of each type, i.e. an optional Suit of Armour replaces a model's default Suit of Armour if applicable. The types of armour below are also referred to as mundane armour:

\subsection{Suits of Armour}
\idx[main=y]{Suits of Armour}

\begin{itemize}
	\item \idx[main=y]{Light Armour}Light Armour: +1 Armour
	\item \idx[main=y]{Heavy Armour}Heavy Armour: +2 Armour
	\item \idx[main=y]{Plate Armour}Plate Armour: +3 Armour
\end{itemize}

\subsection{Shields}
\idx[main=y]{\shield}

\begin{itemize}
	\item Shield: +1 Armour
\end{itemize}

While using a \hyperref[twohanded]{Two-Handed} weapon, a Shield is only used when being attacked by Ranged Attacks.

\begin{table*}[!bph]
	\centering
	\begin{tabular}{r c c c c l}
	  \toprule
	  Weapon & Range & Shots & Strength & Armour Penetration & Attack Attributes \\
	  \idx[main=y]{Bow}\textbf{Bow} & \distance{24} & 1 & 3 & 0 & \hyperref[volley_fire]{Volley Fire} \\
	  \idx[main=y]{Crossbow}\textbf{Crossbow} & \distance{30} & 1 & 4 & 1 & \hyperref[unwieldy]{Unwieldy} \\
	  \idx[main=y]{Handgun}\textbf{Handgun} & \distance{24} & 1 & 4 & 2 & \hyperref[unwieldy]{Unwieldy} \\
	  \idx[main=y]{Longbow}\textbf{Longbow} & \distance{30} & 1 & 3 & 0 & \hyperref[volley_fire]{Volley Fire} \\
	  \idx[main=y]{Pistol}\textbf{Pistol} & \distance{12} & 1 & 4 & 2 & \hyperref[quick_to_fire]{Quick to Fire} \\
	  \idx[main=y]{Throwing Weapons}\textbf{Throwing Weapons} & \distance{8} & 2 & as user & as user & \hyperref[accurate]{Accurate}, \hyperref[quick_to_fire]{Quick to Fire}  \\
	  \bottomrule
	\end{tabular}
	\caption{Mundane Shooting Weapons.}
	\label{table/shooting_weapons}
\end{table*}

\section{Weapons}
\idx[main=y]{Weapons}\label{weapons}

Weapons are divided into three categories: Close Combat Weapons, Shooting Weapons, and Artillery Weapons. The weapons listed in the following pages are also referred to as mundane weapons.

\subsection{Close Combat Weapons}
\idx[main=y]{Close Combat Weapons}\label{close_combat_weapons}

Close Combat Weapons are used in close combat and can confer various benefits and drawbacks to the bearer while using the weapon. Mundane Close Combat Weapons are listed below.

\paragraph{\gw}
\idx[main=y]{\gw}

\hyperref[twohanded]{\textbf{Two-Handed}}. Attacks made with a Great Weapon gain +2 Strength, +2 Armour Penetration, and  strike at \hyperref[initiative_order]{Initiative Step} 0 (regardless of the wielder's Agility).

\paragraph{\halberd}
\idx[main=y]{\halberd}

\hyperref[twohanded]{\textbf{Two-Handed}}. Attacks made with a Halberd gain +1 Strength and +1 Armour Penetration.

\paragraph{\hw}
\idx[main=y]{\hw}

All models come equipped with a Hand Weapon as their default equipment. If a model has any Close Combat Weapon other than a Hand Weapon, it cannot choose to use the Hand Weapon, unless specifically stated otherwise. Models on foot wielding a Hand Weapon alongside a Shield gain \hyperref[parry]{\textbf{Parry}}.

\paragraph{\lance}
\idx[main=y]{\lance}

Attacks made with a Lance and allocated towards models in the wielder's Front Facing gain \hyperref[devastating_charge]{\textbf{Devastating Charge (+2 Strength, +2 Armour Penetration)}}. Infantry cannot use Lances.

\paragraph{\lightlance}
\idx[main=y]{\lightlance}

Attacks made with a Light Lance and allocated towards models in the wielder's Front Facing gain \hyperref[devastating_charge]{\textbf{Devastating Charge (+1 Strength, +1 Armour Penetration)}}. Infantry cannot use Light Lances.

\paragraph{\pw}
\idx[main=y]{\pw}

\hyperref[twohanded]{\textbf{Two-Handed}}. The wielder gains +1 Attack Value and +1 Offensive Skill when using Paired Weapons. Attacks made with Paired Weapons ignore \hyperref[parry]{Parry} (while Paired Weapons are often modelled as two Hand Weapons, they are considered a separate weapon category for rules purposes).

\paragraph{\spear}
\idx[main=y]{\spear}

Attacks made with a Spear gain \hyperref[fight_in_extra_rank]{\textbf{Fight in Extra Rank}} and +1 Armour Penetration. In addition, unless the attacking model’s unit is Charging or is Engaged in any Flank or Rear Facing, attacks made with a Spear gain +2 Agility and an additional +1 Armour Penetration in the First Round of Combat. Only Infantry can use Spears.

\subsubsection{Choosing a Close Combat Weapon}

If a model has more than one Close Combat Weapon, it must choose which one to use in the First Round of Combat, at step 2 of the \hyperref[round_of_combat_sequence]{Round of Combat Sequence}. It must then continue to use the same weapon for the duration of that combat. All \rnf{} models in a unit must \textbf{always} choose the same Close Combat Weapon, unless they are forced to use enchanted weapons.

\columnbreak

\subsection{Shooting Weapons}
\idx[main=y]{\shootingweapons}\label{shooting_weapons}

Shooting Weapons are used for making Shooting Attacks. Each model part can normally only use one Shooting Weapon per phase even if it is equipped with more than one, and all non-Champion \rnf{} models in a unit must use the same Shooting Weapon. Each Shooting Weapon has a maximum range, a number of shots fired, a Strength, and an Armour Penetration value, and can have one or more \hyperref[attack_attributes]{Attack Attributes}. Attack Attributes listed for a Shooting Weapon only apply to the Shooting Attacks made with that weapon. Mundane Shooting Weapons are listed in table \ref{table/shooting_weapons}.

\subsection{Artillery Weapons}
\idx[main=y]{\artilleryweapons}\label{artillery_weapons}

Artillery Weapons are a special kind of Shooting Weapon. These weapons are often installed on \hyperref[war_machine]{War Machines}, but can on other occasions be carried by \hyperref[chariot]{Chariots} or Gigantic Beasts or contained within \hyperref[special_items]{Special Items}.

Artillery Weapons are Shooting Weapons that always have the \hyperref[reload]{\textbf{Reload!}} Attack Attribute, and they may have specific profiles for range, shots, Strength, Armour Penetration, and other Attack Attributes, which you will find in their description. Some Artillery Weapons may have further rules as detailed below.

\subsubsection{Cannon}
\idx[main=y]{\cannon}\label{cannon}

Cannon attacks ignore to-hit modifiers from Soft Cover and Hard Cover. They gain a +1 to-hit modifier when targeting units consisting entirely of models of Gigantic Height that do not benefit from Cover. On a natural to-hit roll of \result{1}, a Cannon Misfires: roll on the \hyperref[the_misfire_table]{Misfire Table} (table \ref{table/misfire}, page \pageref{table/misfire}) and apply the corresponding result (a to-hit roll resulting in a Misfire cannot be rerolled).

\subsubsection{Catapult (X\timess{}Y)}
\idx[main=y]{\catapult{}}\label{catapult}

Catapult attacks ignore to-hit modifiers from Soft Cover and Hard Cover. Resolve Catapult attacks as follows:

\begin{itemize}
	\item On a natural to-hit roll of \result{1}, it Misfires: roll on the \hyperref[the_misfire_table]{Misfire Table} (table \ref{table/misfire}, page \pageref{table/misfire}) and apply the corresponding result (a to-hit roll resulting in a Misfire cannot be rerolled).
	\item On a successful to-hit roll, the attack gains \hyperref[area_attack]{Area Attack (X\timess{}Y)}. Resolve the attack with the Strength and Armour Penetration stated in the Catapult's description.
	\item On any other to-hit result, roll to hit with a new Catapult attack, referred to as Partial Hit, and ignore any Misfire. If it hits, this attack gains \hyperref[area_attack]{Area Attack (X\timess{}Y)}, but you reduce both X and Y by 1. If either value reaches 0, no hits are inflicted. All hits are at half Strength and half Armour Penetration, rounding fractions up. In addition, the attack loses all benefits from the Strength, Armour Penetration, and/or Attack Attributes written in square brackets (if any; see \hyperref[area_attack]{Area Attack}). If it misses, no further attack can be generated this way.
\end{itemize}

\subsubsection{Flamethrower}
\idx[main=y]{\flamethrower}\label{flamethrower}

Flamethrowers do not roll to hit. Instead, roll a D6 (this is not considered a to-hit roll). On a natural roll of \result{1}, it Misfires: roll on the \hyperref[the_misfire_table]{Misfire Table} (table \ref{table/misfire}, page \pageref{table/misfire}) with a \minuss{}1 modifier and apply the corresponding result. On any other natural result the attack is successful.

Determine which Arc of the target the attacker is Located in:
\begin{itemize}
	\item If the attacker is Located in the Front or the Rear Arc, the attack causes D6 hits, +D3 hits for each rank after the first up to a maximum of +4D3.
	\item If the attacker is Located in either Flank Arc, the attack causes D6 hits, +D3 hits for each file after the first up to a maximum of +4D3.
\end{itemize}
The total number of hits cannot exceed the number of models in the unit.

Some Flamethrowers have a higher Strength, Armour Penetration, and/or additional Attack Attributes stated in curly brackets (such as Strength 4 \{5\}, Armour Penetration 1 \{2\}, \{\hyperref[multiple_wounds]{Multiple Wounds (D3)}\}). If so, use the Strength, Armour Penetration, and Attack Attributes in curly brackets when shooting at a target within Short Range.

\subsubsection{Volley Gun}
\idx[main=y]{\volleygun{}}\label{volley_gun}

The number of shots fired by a Volley Gun is a random number. When rolling for the number of shots for a Volley Gun attack, if a single natural \result{6} is rolled (after any reroll), this attack suffers a \minuss{}1 to-hit modifier; instead, if two or more natural \result{6} are rolled, the attack fails and the Volley Gun Misfires: roll on the \hyperref[the_misfire_table]{Misfire Table} (table \ref{table/misfire}, page \pageref{table/misfire}) and apply the corresponding result.

\subsection{Misfire Table}
\idx[main=y]{Misfire Table}\label{the_misfire_table}

A to-hit roll resulting in a Misfire cannot be rerolled. When an Artillery Weapon Misfires, roll a D6 and consult table \ref{table/misfire} below (a result of \result{0} or less may happen when there is a negative modifier to the roll, as for \hyperref[flamethrower]{Flamethrowers}).

\begin{Figure}
	\Tanchor
	\centering
	\setlength{\tabcolsep}{3pt}
	\begin{tabular}{@{}M{1.5cm}m{\columnwidth-1.5cm-6pt}@{}}
		\textbf{Result} & \centering\textbf{Misfire Effect} \tabularnewline
		\midrule
		\textbf{0\newline (or less)} & \textbf{Explosion!} All models within \distance{D6} of the Misfiring model suffer a hit with Strength 5 and Armour Penetration 2. These hits are not considered to be made with the Shooting Weapon. The shooting model is then destroyed and removed as a casualty. \tabularnewline
		\textbf{1--2} & \textbf{Breakdown.} The model cannot shoot with the weapon for the rest of the game. \tabularnewline
		\textbf{3--4} & \textbf{Jammed.} The Artillery Weapon may not shoot in the owner's next Player Turn. If the model is a \hyperref[war_machine]{War Machine}, the model is \hyperref[shaken]{Shaken} until the end of the owner's next Player Turn instead.\tabularnewline
		\textbf{5+} & \textbf{Malfunction.} The shooting model loses a Health Point with no saves of any kind allowed. \tabularnewline
		\bottomrule
	\end{tabular}
	\caption{Misfire Table.}
	\label{table/misfire}
\end{Figure}

\section{Attack Attributes}
\idx[main=y]{Attack Attributes}\label{attack_attributes}

Attack Attributes can be given to a model part, to a weapon, to a spell, or to a \hyperref[special_attacks]{Special Attack}. Remember that an Attack Attribute that is given to a unit is automatically given to every model in that unit (see \totalref{units}), and if it is given to a model, it is automatically given to all model parts of that model (see \totalref{multipart_models}).

Attack Attributes given to a weapon, spell, or Special Attack are always applied to the attacks made with that weapon, spell, or Special Attack. They are however not applied to any other attacks made by the corresponding model part.

Example: A Character on a Chariot has a Weapon Enchantment that makes the attacks made with its Hand Weapon Divine Attacks and Magical Attacks. This means that only the Close Combat Attacks made with the enchanted weapon become Divine Attacks and Magical Attacks, while all other attacks made by the model (e.g. \hyperref[impact_hits]{Impact Hits} and Close Combat Attacks from the horses pulling the Chariot) are not affected.

Attack Attributes are divided into the following sub-categories that define which attacks they affect when given to a model part. Note that these sub-categories are irrelevant for Attack Attributes given to a weapon, spell, or an attack, including Special Attacks (so an Attack Attribute with the Close Combat keyword can be given e.g. to a Shooting Weapon).

\paragraph{Close Combat}
\idx[main=y]{Close Combat (Attack Attribute)}

A model part with an Attack Attribute with this keyword applies the rules of the Attack Attribute to all its Close Combat Attacks.

\paragraph{Shooting}
\idx[main=y]{Shooting (Attack Attribute)}

A model part with an Attack Attribute with this keyword applies the rules of the Attack Attribute to all its Shooting Attacks that are not \hyperref[special_attacks]{Special Attacks}.

Example: A unit is the target of a spell that makes the unit's attacks Divine Attacks and Flaming Attacks. Divine Attacks has the keyword Close Combat, while Flaming Attacks has the keywords Close Combat and Shooting. This means that the Close Combat Attacks of all model parts of that unit are now both Divine Attacks and Flaming Attacks, while the Shooting Attacks of all model parts of the unit are only Flaming Attacks. Any Special Attacks (including Special Attacks that are Shooting Attacks) or spells from the unit are not affected.

\subsection{Conditional Application}
\label{conditional_application}

Attack Attributes may only work against certain enemies, which are then stated in brackets after \enquote{against}. There may already be some piece of information relative to the rule specified between brackets, as in \hyperref[multiple_wounds]{Multiple Wounds~(2)}. In this case the conditions for the rule to work are written in the same brackets, after a comma. This can e.g. be all models from a given Army Book, with a given Model Rule, of a given Height, or of a given Type.

If the Attack Attribute is effective against more than one type of enemy, they are separated by commas. If no comma but instead \enquote{and} is used, this means that the rule works only against enemies that fulfil all criteria. For example, Multiple Wounds~(2, against Large and Beast, Gigantic) means that Multiple Wounds can be used against models that are both Large and Beast, as well as against models that are Gigantic, regardless of the models' Type.

Attack Attributes with Conditional Application can only be applied when the affected attacks are either allocated towards or distributed onto a Health Pool where all models fulfil the requirements.

\subsection{List of Attack Attributes}
\label{list_of_attack_attributes}

\subsubsection{Accurate -- Shooting}
\idx[main=y]{\accurate}\label{accurate}

The attack doesn't suffer the \minuss{}1 to-hit modifier for shooting at \hyperref[long_range]{Long Range}.

\newcommand{\figAATotalA}{3 + 3 + 3 = 9 Hits}
\newcommand{\figAATotalB}{3 + 2 = 5 Hits}
\newcommand{\figAATotalC}{1 Hit}
\newcommand{\figAATotalD}{3 + 3 = 6 Hits}
\newcommand{\figAATotalE}{1 + 1 = 2 Hits}
\newcommand{\figAATotalF}{1 + 1 + 1 + 1 = 4 Hits}
\newcommand{\figAAAreaAttack}[1]{\areaattack{#1}}

\begin{figure*}[!t]
	\centering
	\def\svgwidth{0.8\textwidth}
	\subimport{../pics/}{area_attack.pdf_tex}
	\caption{Examples of Area Attacks.\captionposttitle%
		Units A to D are hit by an Area Attack (3\timess{}3) from an attacker Located in their Front Arc.\captionpar
		In unit A, there are 3 ranks with more than 3 models. Number of hits: 3 + 3 + 3 = 9.\captionpar
		 In unit B, the first rank has more than 3 models, but the second rank only has two models. There is no third rank.
		Number of hits: 3 + 2 = 5.\captionpar
		Since unit C is a single model, there will always only be a single rank with a single model. Number of hits: 1.\captionpar
		In unit D, the first and second rank have 3 models. There is no third rank. Number of hits: 3 + 3 = 6.\captionpar
		Unit E is hit by an Area Attack (1\timess{}4) from an attacker Located in its Front Arc. The first and second rank have more than 1 model. There is no third or fourth rank. Number of hits: 1 + 1 = 2.\captionpar
		Unit F is hit by an Area Attack (1\timess{}4) from an attacker Located in its Flank Arc. There are more than 4 files with 1 or more models. Number of hits: 1 + 1 +1 +1 = 4.%
	}
	\label{figure/area_attack}
\end{figure*}

\subsubsection{Area Attack (X\timess{}Y)}
\idx[main=y]{\areaattack{}}\label{area_attack}

When the attack hits, determine the position of the attacker's base:
\begin{itemize}
	\item In case of a Ranged Attack, determine which Arc of the target the attacker's base is Located in.
	\item In case of a Melee Attack, determine which Facing of the target the attacker is Engaged in.
\end{itemize}

Front or Rear: Choose up to Y different ranks of the target.\newline
Flank: Choose up to Y different files of the target.

For each rank/file selected this way, the unit suffers X hits, to a maximum equal to the number of models in this rank/file. A single Area Attack can never cause more hits than there are models in the unit.

Some Area Attacks have a higher Strength, Armour Penetration, and/or additional Attack Attributes stated in square brackets (e.g. Strength 3 [7], Armour Penetration 0 [4], [\hyperref[multiple_wounds]{Multiple Wounds (D3)}]). If so, a single hit from this attack, chosen by the attacker (following the normal rules for \hyperref[distributing_hits]{Distributing Hits}), uses the Strength, Armour Penetration, and Attack Attributes in brackets. The bracketed values and Attack Attributes are not applied to any other hits.

Figure \ref{figure/area_attack} illustrates examples of different Area Attacks.

\subsubsection{Battle Focus -- Close Combat}
\idx[main=y]{\battlefocus}\label{battle_focus}

If the attack hits with a natural to-hit roll of \result{6}, the attack causes one additional hit (i.e. usually two hits instead of one).

In order for Shooting Attacks using \hyperref[hopeless_shots]{Hopeless Shots} to cause one additional hit, the first to-hit roll must be a natural \result{6} and the second to-hit roll must be successful in order to hit the target.

\subsubsection{Crush Attack -- Close Combat}
\idx[main=y]{\crushattack}\label{crush_attack}

At the end of step 4 of the \hyperref[round_of_combat_sequence]{Round of Combat Sequence} (immediately after issuing and accepting Duels), the model part may declare that it will use its Crush Attack this Round of Combat. It performs a single Close Combat Attack at \hyperref[initiative_order]{Initiative Step} 0, with Strength 10, Armour Penetration 10 (regardless of the user's Agility, Strength, and Armour Penetration), and \hyperref[multiple_wounds]{Multiple Wounds (D3+1)}. The following restrictions apply to Crush Attacks:

\begin{itemize}
	\item They cannot be made as \hyperref[supporting_attacks]{Supporting Attacks}.
	\item They never benefit from any weapons or other Attack Attributes the model part may have.
	\item The model part cannot make any other Close Combat Attacks during this Round of Combat (including other Crush Attacks, but can still use its Special Attacks such as Stomp Attacks or Impact Hits).
\end{itemize}

\subsubsection{Devastating Charge (X) -- Close Combat}
\idx[main=y]{\devastatingcharge{}}\label{devastating_charge}

A Charging model part with Devastating Charge, or using a weapon with Devastating Charge, gains the Model Rules and Characteristic modifiers stated in brackets.

For example, a model part with Devastating Charge (+1 Strength, \hyperref[poison_attacks]{Poison Attacks}) gains +1 Strength and Poison Attacks when it is Charging.

Remember that Special Attacks cannot be affected by Attack Attributes, so the Model Rules and Characteristic modifiers gained from Devastating Charge are not applied to Special Attacks, like \hyperref[impact_hits]{Impact Hits} and \hyperref[stomp_attacks]{Stomp Attacks}, unless specifically stated otherwise.

This rule is cumulative: a model part with several instances of Devastating Charge applies all Attack Attributes and Characteristic modifiers from all of them when Charging.

\subsubsection{Extra Support (X) -- Close Combat}
\idx[main=y]{\extrasupport{}}\label{extra_support}

The model part ignores the general Attack Value modifier for Supporting Attacks, but its Attack Value is never higher than the value stated in brackets (X) while performing Supporting Attacks.

\subsubsection{Divine Attacks -- Close Combat}
\idx[main=y]{\divineattacks}\label{divine_attacks}

Successful \hyperref[aegis]{Aegis} Saves taken against the attack must be rerolled.

\subsubsection{Fight in Extra Rank -- Close Combat}
\idx{Supporting Attacks}\idx[main=y]{\fightinextrarank}\label{fight_in_extra_rank}

Model parts with Fight in Extra Rank, or using a weapon with Fight in Extra Rank, can make \hyperref[supporting_attacks]{Supporting Attacks} from an additional rank (normally, this means that models with Fight in Extra Rank will be able to make Supporting Attacks from the third rank). This rule is cumulative, allowing an additional rank to make Supporting Attacks for each instance of Fight in Extra Rank.

\subsubsection[Flaming Attacks -- Close Combat, Shooting]{Flaming Attacks -- Close Combat,\\ Shooting}
\idx[main=y]{\flamingattacks}\label{flaming_attacks}

The attack ignores \hyperref[fortitude]{Fortitude} Saves and must reroll failed to-wound rolls against models with \hyperref[flammable]{Flammable}.

\subsubsection{Harnessed -- Close Combat, Shooting}
\idx[main=y]{\harnessed}\label{harnessed}

Model parts with Harnessed cannot make \hyperref[supporting_attacks]{Supporting Attacks} and cannot use weapons. Shooting Weapons carried by model parts with Harnessed can be used by other model parts of the same model (as long as they do not have Harnessed or \hyperref[inanimate]{Inanimate}). A model with at least one model part with Harnessed is considered to be mounted.

\subsubsection{Hatred -- Close Combat}
\idx[main=y]{\hatred}\label{hatred}

During the First Round of Combat, failed to-hit rolls from attacks with Hatred must be rerolled.

\subsubsection{Inanimate -- Close Combat, Shooting}
\idx[main=y]{\inanimate}\label{inanimate}

Model parts with Inanimate cannot make Close Combat Attacks and cannot use Shooting Weapons. Shooting Weapons carried by model parts with Inanimate can be used by other model parts of the same model (as long as they do not have \hyperref[harnessed]{Harnessed} or Inanimate).

\subsubsection{Lethal Strike -- Close Combat}
\idx[main=y]{\lethalstrike}\label{lethal_strike}

An attack with Lethal Strike that wounds with a natural to-wound roll of \result{6} has its Armour Penetration \textbf{always} set to 10 and ignores \hyperref[fortitude]{Fortitude} Saves.

\subsubsection{Lightning Reflexes -- Close Combat}
\idx[main=y]{\lightningreflexes}\label{lightning_reflexes}

A Close Combat Attack with Lightning Reflexes gains a +1 to-hit modifier. Model parts with this Attack Attribute wielding Great Weapons do not gain this +1 to-hit modifier, but strike with the Great Weapon at the \hyperref[initiative_order]{Initiative Step} corresponding to their Agility instead of always striking at Initiative Step 0.

\subsubsection[Magical Attacks -- Close Combat, Shooting]{Magical Attacks -- Close Combat,\\ Shooting}
\idx[main=y]{\magicalattacks}\label{magical_attacks}

The Attack Attribute doesn't confer any additional effects. However, the Attack Attribute interacts with other rules, such as \hyperref[aegis]{Aegis} (X, against Magical Attacks).

\subsubsection{March and Shoot -- Shooting}
\idx[main=y]{\marchandshoot}\label{march_and_shoot}

March Moving in the same Player Turn while affected by this Attack Attribute does not prevent the attack from being performed, unless the attack is also subject to Move or Fire.

\subsubsection{Move or Fire -- Shooting}
\idx[main=y]{\moveorfire}\label{move_or_fire}

The attack may not be used if the attacking model has made an \hyperref[advance_move]{Advance Move}, \hyperref[march_move]{March Move}, \hyperref[random_movement]{Random Movement}, \hyperref[reform]{Reform}, or \hyperref[pivots_and_wheels]{Pivot} during the current Player Turn. Note that the normal limitations still apply (e.g. no shooting after a \hyperref[failed_charge]{Failed Charge Move}).

\subsubsection{Multiple Wounds (X) -- Close Combat}
\idx[main=y]{\multiplewounds{}{}}\label{multiple_wounds}

Unsaved wounds caused by the attack are multiplied into the value given in brackets (X). If the value is a dice (e.g. Multiple Wounds (D3)), roll one dice for each unsaved wound from an attack with Multiple Wounds. The amount of wounds that the attack is multiplied into can never be higher than the Health Points Characteristic of the target (excluding Health Points lost previously in the battle).

For example, if a Multiple Wounds (D6) attack wounds a unit of Trolls (HP 3) and rolls a \result{5} for the multiplier, the number of unsaved wounds is reduced to 3, even if the Troll unit has already lost one or two Health Points previously in battle.

\idx[main=y]{Clipped Wings}
If Clipped Wings is stated after the X value in brackets, any unsaved wound caused by the attack against a model with \hyperref[fly]{Fly} is multiplied into X+1 instead of X.

\subsubsection[Poison Attacks -- Close Combat, Shooting]{Poison Attacks -- Close Combat,\\ Shooting}
\idx[main=y]{\poisonattacks}\label{poison_attacks}

If the attack hits with a natural to-hit roll of \result{6}, it automatically wounds with no to-wound roll needed.

Shooting Attacks using \hyperref[hopeless_shots]{Hopeless Shots} automatically wound only if the first to-hit roll is a natural \result{6}. Note that the second to-hit roll must still be successful in order to hit the target.

If the attack can be turned into more than one hit (e.g. a hit with \hyperref[area_attack]{Area Attack} or \hyperref[battle_focus]{Battle Focus}), only a single hit, chosen by the attacker, automatically wounds. All other hits must roll to wound as normal.

\subsubsection{Quick to Fire -- Shooting}
\idx[main=y]{\quicktofire}\label{quick_to_fire}

The attack doesn't suffer the \minuss{}1 to-hit modifier for Moving and Shooting.

\subsubsection{Rage -- Close Combat}
\idx[main=y]{\rage}\label{rage}

 Whenever the model loses a Health Point, it gains +1 Attack Value. Whenever it gains a Health Point, it suffers \minuss{}1 Attack Value.

\subsubsection{Reload! -- Shooting}
\idx[sortlabel=Reload,main=y]{Reload\!}\label{reload}

The attack cannot be used for a Stand and Shoot Charge Reaction.

\subsubsection{Shoot in Extra Rank -- Shooting}
\idx[main=y]{\shootinextrarank}\label{shoot_in_extra_rank}

The model part may use Shooting Attacks from one additional rank (normally, this means that models with Shoot in Extra Rank will be able to make Shooting Attacks from the third rank). This rule is cumulative, allowing an additional rank to make Shooting Attacks for each instance of Shoot in Extra Rank.

\subsubsection{Steady Aim -- Shooting}
\idx[main=y]{\steadyaim}\label{steady_aim}

The model does not suffer the \minuss{}1 to-hit modifier for Stand and Shoot Charge Reactions.
\subsubsection{Toxic Attacks -- Close Combat}
\idx[main=y]{\toxicattacks}\label{toxic_attacks}

The attack has its Strength \textbf{always} set to 3 and its Armour Penetration \textbf{always} set to 10.

\subsubsection{Two-Handed}
\idx[main=y]{\twohanded}\label{twohanded}

A model using a weapon with Two-Handed can simultaneously use a Shield only when being attacked by Ranged Attacks.

\subsubsection{Unwieldy -- Shooting}
\idx[main=y]{\unwieldy}\label{unwieldy}

The attack suffers an additional \minuss{}1 to-hit modifier for Moving and Shooting (for a total of \minuss{}2). When combined with \hyperref[quick_to_fire]{Quick to Fire}, the attack can only ignore the normal \minuss{}1 to-hit modifier for Moving and Shooting, not the additional \minuss{}1 to-hit modifier from Unwieldy.

\subsubsection{Volley Fire -- Shooting}
\idx[main=y]{\volleyfire}\label{volley_fire}

If at least one model in a unit can draw Line of Sight to the target, then all model parts using Volley Fire in the same unit ignore all intervening models of their own Height or smaller for Line of Sight and Cover purposes.

In addition, unless making a Stand and Shoot Charge Reaction, models in a unit in \hyperref[line_formation]{Line Formation} that has not moved during this Player Turn gain \shootinextrarank{}.

\subsubsection{Weapon Master -- Close Combat}
\idx[main=y]{\weaponmaster}\label{weapon_master}

At step 2 of the Round of Combat Sequence (Choose weapon) of each Round of Combat, model parts with Weapon Master may choose which weapon they fight with. This includes selecting to use a Hand Weapon even if they have other weapons. If armed with a weapon with a Weapon Enchantment, the model part must still use it.

\section{Special Attacks}
\idx[main=y]{\specialattacks}\label{special_attacks}

A model part with Special Attacks can make a special type of attack specified by the corresponding rules. Attacks made using Special Attacks cannot be affected by weapons or \hyperref[attack_attributes]{Attack Attributes}, unless specifically stated otherwise.

\subsection{List of Special Attacks}

\subsubsection{Breath Attack (X)}
\idx[main=y]{\breathattack{}}\label{breath_attack}

A model part with Breath Attack can use it only once during the game. If a model has more than one Breath Attack, it can only use one Breath Attack in a single phase. It can be used either as a ranged Special Attack or as a Melee Attack when Engaged in Combat.
\begin{itemize}
	\item As a ranged Special Attack, it follows the rules for a Shooting Attack with \textbf{\marchandshoot}: choose a target using the \hyperref[shooting_with_a_unit]{normal rules} for Shooting Attacks (it can be used for a Stand and Shoot Charge Reaction). The attack has a range of \distance{6}. A model part with both a Breath Attack and a Shooting Weapon can use both in the same Shooting Phase, however only against the same target.
	\item As a Special Attack when Engaged in Combat: the attack is made at the model part's Agility. Declare that you are using the Breath Attack at the start of the Initiative Step (before rolling to hit), and choose a unit in base contact as a target.
\end{itemize}
No matter if it is used as a ranged or Melee Attack, the target of the Breath Attack suffers 2D6 hits. The Strength, Armour Penetration, and Attack Attributes (if any) of these hits are given within brackets, such as in Breath Attack (\St{} 4, \AP{} 1, \hyperref[flaming_attacks]{Flaming Attacks}). If several models in the same unit have this Special Attack, roll for the number of hits separately for each model.

\subsubsection{Grind Attacks (X)}
\idx[main=y]{\grindattacks{}}\label{grind_attacks}

A model part with Grind Attacks resolves these attacks at its Agility. It must choose an enemy unit in base contact with it. The chosen enemy unit suffers a number of hits equal to the value stated in brackets (X). These hits are resolved with the model part's Strength and Armour Penetration.

If a model has both Grind Attacks and \hyperref[impact_hits]{Impact Hits}, it may only use one of these rules in the same Round of Combat (the owner may choose which). If several model parts in a unit have Grind Attacks and if X is a random number (e.g. Grind Attacks (2D3)), roll for the number of hits separately for each model part.

\subsubsection{Impact Hits (X)}
\idx[main=y]{\impacthits{}}\label{impact_hits}

At \hyperref[initiative_order]{Initiative Step} 10, a Charging model part with Impact Hits must choose an enemy unit that is in base contact with the attacking model's Front Facing. This unit suffers a number of hits equal to the value stated in brackets (X). These hits are resolved with the attacking model part's Strength and Armour Penetration.

If a model has both \hyperref[grind_attacks]{Grind Attacks} and Impact Hits, it may only use one of these rules in the same Round of Combat (the owner may choose which). In case of Multipart Models, only model parts that also have \hyperref[harnessed]{Harnessed} or \hyperref[inanimate]{Inanimate} can use Impact Hits. If several models in a unit have Impact Hits, and if X is a random number (e.g. Impact Hits (D6)), roll for the number of hits separately for each model part.

\subsubsection{Stomp Attacks (X)}
\idx[main=y]{\stompattacks{}}\label{stomp_attacks}

At \hyperref[initiative_order]{Initiative Step} 0, a model part with Stomp Attacks must choose an enemy model of Standard Height in base contact with it. The chosen model's unit suffers a number of hits equal to the value stated in brackets (X). These hits can only be distributed onto models of Standard Height (ignore models of a different Height when distributing hits). They are resolved with the model part's Strength and Armour Penetration.

In case of Multipart Models, only model parts that also have \hyperref[harnessed]{Harnessed} can use Stomp Attacks. If several models in a unit have this Special Attack, and if X is a random number (e.g. Stomp Attacks (D6)), roll for the number of hits separately for each model part.

\subsubsection{Sweeping Attack}
\idx[main=y]{\sweepingattack}\label{sweeping_attack}

This attack may be used by units containing models with Sweeping Attack. When the unit \hyperref[advance_move]{Advance Moves} or \hyperref[march_move]{March Moves}, you may nominate a single unengaged enemy unit that the unit with Sweeping Attack moved through or over during this move (meaning their Unit Boundaries were overlapping, even partially). The whole unit makes the Sweeping Attack against the nominated enemy unit, which is resolved when the March or Advance Move is completed. Follow the description in the unit entry. These attacks hit automatically and count as ranged Special Attacks. Each Sweeping Attack can only be performed once per Player Turn.

\RBemc
